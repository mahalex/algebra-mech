\section{Теория групп}

\subsection{Определения и примеры}

\literature{[F], гл.~I, \S~3, п. 1, гл.~X, \S~1, пп. 1--2, \S~5, п. 1;
[K1], гл. 4, \S~2, п. 1; [vdW], гл. 2, \S~6; [Bog], гл. 1, \S~1.}

Мы уже встречали определение группы (см. определение \ref{def_group}):
\begin{definition}\label{def_group_new}
Множество $G$ с бинарной операцией $\circ\colon G\times G\to G$
называется
\dfn{группой}\index{группа}, если выполняются следующие свойства:
\begin{itemize}
\item $a\circ (b\circ c)=(a\circ b)\circ c$ для всех $a,b,c\in G$;
  (\dfn{ассоциативность}\index{ассоциативность!в группе});
\item существует элемент $e\in G$ (\dfn{единичный
    элемент}\index{единичный элемент!в группе}) такой, что
  для любого $a\in G$
  выполнено $a\circ e=e\circ a=a$;
\item для любого $a\in G$ найдется элемент $a^{-1}\in G$ (называемый
  \dfn{обратным}\index{обратный элемент!в группе} к $a$) такой, что
  $a\circ a^{-1}=a^{-1}\circ a=e$.
\end{itemize}
Группа $G$ называется \dfn{коммутативной}, или
\dfn{абелевой}\index{группа!коммутативная}\index{группа!абелева}, если
$a\circ b=b\circ a$ для всех $a,b\in G$.
\end{definition}

В прошлом семестре мы некоторое время изучали {\em группу
  перестановок} $S(X)$ множества $X$
(см. определение~\ref{def:symmetric_group}):
\begin{definition}\label{def:symmetric_group_new}
Множество всех биекций из $X$ в $X$ обозначается через $S(X)$ и
называется \dfn{группой перестановок}\index{группа!перестановок}
множества $X$. Тождественное
отображение $\id_X\colon X\to X$ называется \dfn{тождественной
  перестановкой}\index{тождественная перестановка}.
Если $X=\{1,\dots,n\}$, мы обозначаем группу $S(X)$ через $S_n$ и
называем ее \dfn{симметрической группой на $n$
  элементах}\index{группа!симметрическая}.
\end{definition}
В разделе~\ref{subsect:permutations} мы видели, что группа $S_n$
не является абелевой при $n\geq 3$.

На самом деле мы встречали и другие группы.

\begin{examples}\label{examples:group}
\hspace{1em}
\begin{enumerate}
\item Пусть $R$~--- кольцо (см.определение~\ref{def:ring}). В
  частности, это
  означает что на $R$ задана операция сложения. Из определения кольца
  сразу следует, что $R$ относительно этой операции сложения является
  абелевой группой. Она называется \dfn{аддитивной группой
    кольца}\index{группа!кольца, аддитивная}. В
  частности, множества $\mb Z$, $\mb Q$, $\mb R$, $\mb C$ являются
  абелевыми группами относительно сложения.
\item Пусть $V$~--- векторное пространство над полем $k$
  (см. определение~\ref{def:vector_space}). В частности, на $V$ задана
  операция сложения. Относительно этой операции множество $V$ является
  абелевой группой.
\item\label{item:group_of_units_of_a_field}
  Пусть $k$~--- поле. Тогда умножение является ассоциативной,
  коммутативной операцией, единица поля является нейтральным элементом
  относительно этой операции, и у каждого ненулевого элемента имеется
  обратный. Это означает, что $k^* = k\setminus\{0\}$ является
  абелевой группой. Эта группа называется \dfn{мультипликативной
    группой поля $k$}\index{группа!поля, мультипликативная}. В
  частности, множества $\mb Q^*$, $\mb R^*$, $\mb C$ являются
  абелевыми группами относительно умножения.
\item\label{item:group_of_units} Более общо, пусть $R$~---
  ассоциативное кольцо с единицей (не
  обязательно коммутативное). Обозначим через $R^*$ множество
  {\em двусторонне обратимых} элементов $R$, то есть, множество
  элементов $x\in R$ таких, что существует $y\in R$, для которого
  $xy=yx=1$. Нетрудно проверить (сделайте это!), что множество $R^*$
  образует группу относительно умножения. Эта группа называется
  \dfn{группой обратимых элементов кольца $R$}\index{группа!обратимых
    элементов кольца}. В частности, если $R$~--- поле, то все
  ненулевые элементы $R$ [двусторонне] обратимы, и мы получаем
  мультипликативную группу поля из предыдущего примера. Простейший
  пример: $\mb Z^* = \{1,-1\}$.
\item Пусть $k$~--- некоторое поле, $n\geq 1$. Мы знаем, что множество
  квадратных матриц размера $n\times n$ образует кольцо относительно
  операций сложения и умножения матриц
  (см. замечание~\ref{rem:matrix_multiplication_properties}). Группа
  обратимых элементов этого кольца обозначается через $\GL(n,k)$ и
  называется \dfn{полной линейной группой}\index{группа!полная
    линейная}. Таким образом, $\GL(n,k)$ состоит из обратимых матриц
  размера $n\times n$, и это группа относительно операции умножения.
  В частности, при $n=1$ получаем группу $k^*$ обратимых элементов
  поля $k$ (см. пример~\ref{item:group_of_units_of_a_field}).
\item\label{item:special_linear_example} В продолжение предыдущего
  примера, рассмотрим подмножество
  $\SL(n,k)\subseteq\GL(n,k)$, состоящее из матриц с определителем
  $1$. Напомним, что определитель произведения матриц равен
  произведению их определителей, и
  (см. теорему~\ref{thm:determinant_product}). Более того, если
  $x\in\SL(n,k)$~--- матрица с определителем $1$, то и обратная
  матрица $x^{-1}$ имеет определитель $1$. Поэтому
  множество $\SL(n,k)$ само является группой относительно операции
  умножения. Эта группа называется \dfn{специальной линейной
    группой}\index{группа!специальная линейная}.
\item\label{item:group_of_angles}
  Пусть $\mb T = \{z\in\mb C\mid |z| = 1\}$~--- множество
  комплексных чисел с модулем $1$. Это группа по умножению
  (поскольку модуль комплексного числа мультипликативен,
  см. предложение~\ref{prop_abs_properties}).
  Она часто называется \dfn{группой углов}\index{группа!углов}.
  Ниже
  (см.~пример~\ref{examples:quotient-groups}~(\ref{item:angles-as-quotient-group}))
  мы приведем другое ее описание, не использующее
  комплексных чисел.
\item\label{item:geometric_groups} Наиболее архетипичный пример группы
  выглядит так: рассмотрим все обратимые преобразования
  ({\it автоморфизмы}) некоторого объекта в себя (и/или сохраняющих
  {\it нечто}). Это группа
  относительно композиции: действительно, композиция преобразований
  объекта в себя (сохраняющих {\it нечто}) является преобразованием
  объекта в себя (сохраняющим {\it нечто}); композиция преобразований
  всегда ассоциативна; тождественное преобразование должно сохранять
  {\it нечто} и потому является нейтральным элементом; наконец, мы
  потребовали обратимость, поэтому и с обратными элементами нет
  проблемы. Рассмотренные выше примеры все сводятся к
  этому. Симметрическая группа~--- это просто группа обратимых
  преобразований {\it множества} без всякой дополнительной
  структуры. $\GL(n,k)$~--- группа преобразований векторного
  пространства (сохраняющих структуру векторного пространства~---
  сложение и умножение на скаляры~--- то есть,
  {\it линейных}). $\SL(n,k)$~--- группа линейных преобразований
  определителя $1$, то есть, {\it сохраняющих ориентированный объем}
  (мы узнаем, что это такое, в главе 11). Даже группу целых чисел по
  сложению можно интерпретировать схожим образом: рассмотрим целое
  число $x$ как сдвиг вещественной прямой (с отмеченными целыми
  точками) на $x$ вправо (если $x$ отрицательно, получаем сдвиг
  влево). Композиция таких сдвигов в точности соответствует сложению
  целых чисел. Такой {\it геометрический взгляд} на теорию групп
  чрезвычайно продуктивен: более того, Давид Гильберт
  продемонстрировал, что синтетическая геометрия (эвклидова, геометрия
  Лобачевского, проективная) целиком вкладывается в теорию групп.
\end{enumerate}
\end{examples}

\subsection{Подгруппы}

\literature{[F], гл.~X, \S~1, пп. 3--4, \S~3, п. 6; [vdW], гл. 2,
  \S~7; [Bog], гл. 1, \S~1.}

Ситуация, описанная в примере~\ref{examples:group}
(\ref{item:special_linear_example}),
встречается достаточно часто:
\begin{definition}\label{def:subgroup}
Пусть $G$~--- некоторая группа. Подмножество $H\subseteq G$ называется
\dfn{подгруппой}\index{подгруппа} группы $G$, если выполнены следующие
условия:
\begin{enumerate}
\item если $h,h'\in H$, то $h\circ h'\in H$.
\item если $h\in H$, то $h^{-1}\in H$.
\end{enumerate}
Обозначение: $H\leq G$.
\end{definition}
Заметим, что если $H$~--- подгруппа группы $G$, то множество $H$ само
является группой относительно той же операции (точнее, относительно
{\em ограничения} этой операции на $H$).

\begin{examples}
\begin{enumerate}
\item В любой группе $G$ имеются подгруппы $\{e\}\leq G$ и $G\leq G$;
  подгруппа $\{e\}$ называется
  \dfn{тривиальной}\index{подгруппа!тривиальная} и часто обозначается
  через $1$ или $0$ (если групповая операция в $G$ записывается
  мультипликативно или аддитивно, соответственно).
\item Как мы уже видели выше, $\SL(n,k)\leq\GL(n,k)$.
\item Напомним, что все перестановки из $S_n$ делятся на {\em четные}
  и {\em нечетные} (см. определение~\ref{def:permutation_sign}),
  причем произведение четных перестановок четно
  (теорема~\ref{thm:permutation_sign_product}), и обратная к четной
  перестановке четна
  (следствие~\ref{cor:permutation_sign_inverse}). Это означает, что
  множество четных перестановок образует подгруппу в $S_n$. Она
  обозначается через $A_n$ и называется \dfn{знакопеременной
    группой}\index{группа!знакопеременная}.
\item Рассмотрим аддитивную группу целых чисел $\mathbb Z$. Пусть
  $m\in\mb N$. Множество $m\mb Z = \{mx\mid x\in\mb Z\}$ является
  подгруппой в $\mb Z$. Действительно, $mx+my = m(x+y)\in m\mb Z$ и
  $-mx = m(-x)\in m\mb Z$. В частности, $0\mb Z = 0$, $1\mb Z = \mb
  Z$.
  Ниже мы увидим, что любая подгруппа $\mb Z$
  имеет вид $m\mb Z$ для некоторого натурального $m$.
\end{enumerate}
\end{examples}

\begin{theorem}\label{thm:subgroups_of_z}
Любая подгруппа $G$ аддитивной группы $\mb Z$ целых чисел имеет вид
$m\mb Z$ для некоторого натурального $m$.
\end{theorem}
\begin{proof}
Если $G=\{0\}$, можно взять $m=0$. В противном случае выберем
наименьший по модулю элемент из $G\setminus\{0\}$. Заменив при
необходимости знак, можно считать, что этот элемент больше
нуля. Обозначим его через $m$ и покажем, что $G = m\mb Z$. Во-первых,
для натурального $x$ имеем $mx = \underbrace{m+\dots+m}_{x}\in G$ и
$m(-x) = (-m)x = \underbrace{(-m) + \dots + (-m)}_{x}\in G$; поэтому
$m\mb Z\subseteq G$. Обратно, пусть $g\in G$. Поделим с остатком $g$
на $m$: $g = mq + r$. При этом $0\leq r < |m| = m$. Поскольку $g\in G$
и $mq\in G$, получае, что $r = g - mq\in G$. Если $r\neq 0$, это
противоречит минимальности $m$. Значит, $g = mq$ и мы показали, что
$g\in m\mb Z$. Это доказывает обратное включение $G\subseteq m\mb Z$.
\end{proof}

Полезно знать, что пересечение произвольного (конечного или
бесконечного) набора подгрупп группы $G$ снова является подгруппой в
$G$.
\begin{lemma}\label{lem:intersection_of_subgroups}
Пусть $\{H_i\}_{i\in I}$~--- семейство подгрупп группы $G$.
Обозначим $H=\bigcap_{i\in I} H_i$. Тогда $H\leq G$.
\end{lemma}
\begin{proof}
Если $h,h'\in H$, то $h,h'\in H_i$ и $h^{-1}\in H_i$ для всех $i\in
I$, и поэтому $hh', h^{-1}\in H_i$ для всех $i\in I$, откуда $hh',
h^{-1}\in H$.
\end{proof}

Весьма важен следующий способ построения подгрупп: пусть $X$~---
произвольное {\it подмножество} группы $G$. Мы хотим
<<наименьшими усилиями>> расширить $X$ так, чтобы получилась
подгруппа.

\begin{definition}\label{def:subgroup_spanned}
Пусть $X\subseteq G$~--- подмножество группы $G$. Наименьшая
подгруппа в $G$, содержащая $X$, называется \dfn{подгруппой,
  порожденной подмножеством $X$}\index{подгруппа!порожденная
  подмножеством}, и обозначается через $\la X\ra$. Более подробно,
$\la X\ra\leq G$~--- такая подгруппа группы $G$, что
$X\subseteq \la X\ra$ и для любой подгруппы $H\leq G$, содержащей $X$,
выполнено $\la X\ra\leq H$.
\end{definition}

\begin{remark}
Для конечного множества $X=\{x_1,\dots,x_n\}$ мы часто пишем
$\la x_1,\dots,x_n\ra$ вместо $\la \{x_1,\dots,x_n\}\ra$.
\end{remark}

Определение~\ref{def:subgroup_spanned} хорошо всем, кроме одного: a
priori совершенно не
очевидно, что для данного подмножества $X\subseteq G$ существует
подгруппа $\la X\ra\leq G$ с указанными удивительными свойствами.
Следующее предложение показывает, что это действительно так.
\begin{proposition}\label{prop:subgroup_spanned_as_intersection}
Пусть $G$~--- группа, $X\subseteq G$. Пересечение всех подгрупп в $G$,
содержащих $X$, является подгруппой в $G$, порожденной множеством $X$.
\end{proposition}
\begin{proof}
По лемме~\ref{lem:intersection_of_subgroups} пересечение всех подгрупп
в $G$, содержащих $X$, является подгруппой в $G$. Обозначим ее через
$\la X\ra$ и проверим, что она удовлетворяет
определению~\ref{def:subgroup_spanned}. Действительно, множество $X$
содержится во всех пересекаемых подгруппах, поэтому содержится в
$\la X\ra$. С другой стороны, если $H\leq G$ содержит $X$, то $H$
является одной из пересекаемых подгрупп, поэтому полученное
пересечение $\la X\ra$ содержится в $H$.
\end{proof}

\begin{remark}
Обратите внимание на сходство
предложения~\ref{prop:subgroup_spanned_as_intersection} и определения
линейной оболочки~\ref{dfn:linear-combination-and-span}. Понятие подгруппы,
порожденной множеством элементов $G$, является точным аналогом понятия
линейной оболочки множества элементов векторного
пространства.
\end{remark}

\begin{lemma}
Пусть $G$~--- группа, $X\subseteq G$. Подгруппа, порожденная
множеством $X$~--- это множество всех произведений элементов $X$ и
обратных к ним:
$$
\la X\ra = \{y_1y_2\dots y_n\mid y_i\in X\text{ или }y_i^{-1}\in
X\text{ для всех }i=1,\dots,n\}.
$$
\end{lemma}
\begin{proof}
Обозначим правую часть равенства через $Y$. Докажем сначала, что
$Y\subseteq\la X\ra$. Пусть $y = y_1y_2\dots y_n$~--- некоторый
элемент $Y$; мы знаем, что каждый $y_i$ либо является элементом $X$,
либо является обратным к элементу $X$.
Если $H\leq G$~--- произвольная
подгруппа, содержащая $X$, то $H$ содержит и элементы $y_1,\dots,y_n$,
а потому содержит и их произведение $y$. Значит, $y$ лежит в
пересечении всех таких подгрупп $H$, которое равно $\la X\ra$ по
предложению~\ref{prop:subgroup_spanned_as_intersection}.

Для доказательства обратного включения заметим, что множество $Y$ само
является подгруппой в $G$, содержащей множество $X$. В силу
определения~\ref{def:subgroup_spanned} из этого следует, что
$\la X\ra\leq Y$.
\end{proof}

Следующее понятие продолжает эту мысль, вводя аналог
понятия {\it системы образующих} векторного пространства
(см. определение~\ref{dfn:spanning-set}).

\begin{definition}
Говорят, что группа $G$ \dfn{порождается} множеством $X\subseteq G$,
и что $X$~--- \dfn{система порождающих}\index{система порождающих}
(или \dfn{порождающее множество}\index{порождающее множество}) группы
$G$, если $\la X\ra = G$.
\end{definition}

\begin{examples}
\begin{enumerate}
\item Предложение~\ref{prop:product_of_transpositions} в точности
  показывает, что группа $S_n$ порождается множеством всех
  транспозиций, а вместе с
  предложением~\ref{prop_odd_number_of_elementary_transpositions} оно
  означает, что группа $S_n$ порождается множеством всех элементарных
  транспозиций.
\item Группа целых чисел $(\mathbb Z,+)$ порождается одним элементом
  $1$. Действительно, любое натуральное число $n$ является
  суммой $n$ единиц: $n=\underbrace{1+1+\dots+1}_n$, а любое
  отрицательное число $-n$ является суммой $n$ минус единиц:
  $-n = \underbrace{(-1)+(-1)+\dots+(-1)}$.
\end{enumerate}
\end{examples}

\subsection{Классы смежности и нормальные подгруппы}

\literature{[F], гл.~X, \S~1, пп. 5, \S~2; [K3], гл. 1, \S~2, п. 1;
  [vdW], гл. 2, \S\S~8--9; [Bog], гл. 1, \S~2.}

\begin{definition}
Пусть $G$~--- группа, $H\leq G$~--- ее подгруппа, и $g\in
G$. Множество
$$
gH = \{gh\mid h\in H\}
$$
называется \dfn{правым смежным классом элемента $g$ по подгруппе $H$}.
Аналогично, множество
$$
Hg = \{hg\mid h\in H\}
$$
называется \dfn{левым смежным классом элемента $g$ по подгруппе $H$}.
\end{definition}

\begin{proposition}~\label{prop:group_cosets}
Пусть $G$~--- группа, $H\leq G$.
Любые два правых смежных класса по подгруппе $H$ либо не пересекаются,
либо совпадают. Таким образом, группа $G$ разбивается на правые
смежные классы.
Аналогично, любые два левых смежных класса по подгруппе $H$ либо не
пересекаются, либо совпадают. Таким образом, $G$ разбивается на левые
смежные классы.
\end{proposition}
\begin{proof}
Пусть $gH, g'H$~--- два правых смежных класса. Предположим, что они
пересекаются: $x\in gH\cap g'H$. Тогда $x = gh = g'h'$ для некоторых
$h,h'\in H$, откуда $g = g'h'h^{-1}$. Если $y$~--- еще один элемент
$gH$, $y=gh''$, то $y = g'h'h^{-1}h''$, поэтому $y\in
g'H$. Аналогично, если $y\in g'H$, то $y\in gH$. Поэтому $gH =
g'H$. Осталось заметить, что каждый элемент $g\in G$ лежит в некотором
правом смежном классе, хотя бы, $g\in gH$.
Доказательство для левых смежных классов совершенно аналогично.
\end{proof}

Предложение~\ref{prop:group_cosets} чрезвычайно похоже на
теорему~\ref{thm_quotient_set} о разбиении на классы эквивалентности.
Это не случайно: за смежными классами стоят достаточно естественные
отношения эквивалентности.

\begin{definition}
Пусть $G$~--- группа, $H\leq G$. Введем на $G$ отношения $\sim_H$ и
${}_H{\sim}$. Будем говорить, что
$g\sim_Hg'$, если $g^{-1}g'\in H$.
Будем говорить, что $g{}_H{\sim} g'$, если $g'g^{-1}\in H$.
\end{definition}

\begin{lemma}
Отношения $\sim_H$ и ${}_H{\sim}$ являются отношениями эквивалентности;
класс элемента $g\in G$ по отношению $\sim_H$~--- это в точности
правый смежный класс $gH$, а по отношению ${}_H{\sim}$~--- левый смежный
класс $Hg$.
\end{lemma}
\begin{proof}
Мы докажем лемму только для $\sim_H$ и правых смежных классов;
остальное совершенно аналогично.
Проверим рефлексивность, симметричность и транзитивность отношения
$\sim_H$: для $g\in G$ имеем $g^{-1}g=e\in H$, поэтому $g\sim_Hg$.
Если $g\sim_H g'$, то $g^{-1}g'\in H$, поэтому и $g'^{-1}g =
(g^{-1}g')^{-1}\in H$, откуда $g'\sim_H g$. Наконец, если $g\sim_H g'$
и $g'\sim_H g''$, то $g^{-1}g'\in H$ и $g'^{-1}g''\in H$, поэтому и их
произведение $g^{-1}g''=(g^{-1}g')(g'^{-1}g'')\in H$, откуда
$g\sim_Hg''$.

Заметим, что $y\in G$ лежит в классе элемента $g\in G$
тогда и только тогда, когда $g\sim_H y$
(см. определение~\ref{def_equiv_class}). Это равносильно тому, что
$g^{-1}y\in H$, то есть, что $g^{-1}y = h$ для некоторого $h\in
H$. Это, в свою очередь, равносильно тому, что $y=gh$, то есть, что
$y\in gH$.
\end{proof}

\begin{definition}
Пусть $G$~--- группа, $H\leq G$.
Множество правых смежных классов $G$ по $H$ (оно же фактор-множество
$G$ по отношению эквивалентности $\sim_H$) обозначается через
$G/H$. Множество левых смежных классов $G$ по $H$ (оно же
фактор-множество $G$ по отношению эквивалентности ${}_H{\sim}$)
обозначается через $H\backslash G$.
\end{definition}

\begin{remark}\label{rem:coset_analogy}
Отношения $\sim_H$ и ${}_H{\sim}$ являются прямыми аналогами сравнения
по модулю подпространства (см. определение~\ref{def:quotient_space});
однако, отсутствие коммутативности приводит к тому, что необходимо
рассматривать два варианта обобщения: условие $v_1-v_2\in U$ из
определения~\ref{def:quotient_space} мы заменяем на $v_1v_2^{-1}\in U$ в
одном варианте и на $v_2^{-1}v_1\in U$ в другом. Если группа $G$ абелева, то
$gH = Hg$ для всех $g\in G$, и отношения $\sim_H$, ${}_H{\sim}$
совпадают.
\end{remark}

Продолжим аналогию с линейной алгеброй: следующим шагом в построении
фактор-пространства было введение структуры векторного пространства на
множестве классов эквивалентности по модулю подпространства
(предложение~\ref{prop:quotient_space}).
В случае групп отсутствие коммутативности приводит к фатальным
последствиям: оказывается, что для произвольной подгруппы $H\leq G$
фактор-множество $G/H$ не обязано снабжаться естественной структурой
группы. Для того, чтобы $G/H$ оказалось группой, необходимо наложить
на $H$ дополнительное условие {\it нормальности}.

\begin{definition}
Пусть $G$~--- группа. Подгруппа $H\leq G$ называется
\dfn{нормальной}\index{подгруппа!нормальная} (обозначение: $H\trleq
G$), если для любого элемента $g\in G$ его левый и правый смежный
классы совпадают: $Hg = gH$.
\end{definition}

Полезны следующие переформулировки нормальности.

\begin{lemma}\label{lem:normal_subgroup}
Пусть $G$~--- группа, $H\leq G$. Следующие условия
равносильны: 
\begin{enumerate}
\item $H$ нормальна в $G$;
\item $gHg^{-1} = H$ для всех $g\in G$;
\item $gHg^{-1}\subseteq H$ для всех $g\in G$.
\end{enumerate}
(Здесь $gHg^{-1} = \{ghg^{-1}\mid h\in H\}$).
\end{lemma}
\begin{proof}
\begin{itemize}
\item[$1\Rightarrow 2$] Пусть $Hg = gH$ и $h\in H$.
Рассмотрим элемент $ghg^{-1}$. По предположению элемент
$gh$ можно записать в виде $h'g$ для некоторого $h'\in H$.
Поэтому $ghg^{-1} = (gh)g^{-1} = (h'g)g^{-1} = h'\in H$.
Это значит, что $gHg^{-1}\subseteq H$.
Обратно, для $h\in H$ запишем $h = hgg^{-1}$; по предположению элемент
$hg$ можно записать в виде $gh'$ для некоторого $h'\in H$. Значит,
$h = (hg)g^{-1} = gh'g^{-1}\in gHg^{-1}$. Отсюда $H\subseteq
gHg^{-1}$, и необходимое равенство доказано.
\item[$2\Rightarrow 3$] Очевидно.
\item[$3\Rightarrow 1$] Пусть $gHg^{-1}\subseteq H$. Возьмем $h\in H$
  и рассмотрим элемент $gh$. Мы знаем, что $ghg^{-1} = h'\in H$, откуда
  $gh = h'g$; поэтому $gH\subseteq Hg$. Обратно,
  рассмотрим элемент $hg\in Hg$. Применяя предположение к $g^{-1}$,
  получаем, что $g^{-1}Hg\subseteq H$. Значит, элемент $g^{-1}hg=h''$
  лежит в $H$. Отсюда $hg = gh''$, и мы показали, что $Hg\subseteq gH$.
\end{itemize}
\end{proof}

\begin{definition}
Пусть $G$~--- группа, $g,h\in G$. Элемент $ghg^{-1}$ называется
\dfn{сопряженным к $h$ при помощи $g$}; говорят, что элементы $h$ и
$ghg^{-1}$ \dfn{сопряжены}\index{сопряжение!в группе}. Обозначение:
$ghg^{-1} = {}^gh$.
\end{definition}

\begin{remark}
Из замечания~\ref{rem:coset_analogy} следует, что все подгруппы
абелевой группы нормальны.
\end{remark}

\hspace{0em}
\begin{examples}\label{examples:normal_subgroups}
\hspace{1em}
\begin{enumerate}
\item $\SL(n,k)\trleq\GL(n,k)$. Действительно, если $h\in\SL(n,k)$ и
  $g\in\GL(n,k)$, то $\det(ghg^{-1}) =
  \det(g)\cdot\det(h)\cdot\det(g^{-1}) = \det(h) = 1$, поэтому
  ${}^gh\in\SL(n,k)$.
\item $A_n\trleq S_n$. Это доказывается совершенно аналогично
  предыдущему примеру, с заменой определителя на знак
  перестановки. Нормальность в обоих этих примерах также следует из
  леммы~\ref{prop:kernel_and_image}.
\item\label{item:normal_subgroup_of_index_2} Любая подгруппа индекса
  $2$ нормальна. Мы докажем это чуть позже.
\end{enumerate}
\end{examples}

\subsection{Гомоморфизмы групп}

\literature{[F], гл.~X, \S~3, п. 1; [K1], гл. 4, \S~2, пп. 3--4;
  [vdW], гл. 2, \S~10; [Bog], гл. 1, \S~3.}

\begin{definition}
Пусть $G,H$~--- группы.
Отображение $\ph\colon G\to H$ называется \dfn{гомоморфизмом
  групп}\index{гомоморфизм!групп},
если $\ph(xy) = \ph(x)\ph(y)$ для всех $x,y\in G$.
\end{definition}
\begin{lemma}
Пусть $\ph\colon G\to H$~--- гомоморфизм групп. Тогда $\ph(e_G) = e_H$
и $\ph(x^{-1}) = \ph(x)^{-1}$ для всех $x\in G$.
\end{lemma}
\begin{proof}
Заметим, что $e_G\cdot e_G = e_G$. Поэтому $\ph(e_G) = \ph(e_G\cdot
e_G) = \ph(e_G)\cdot \ph(e_G)$. Домножим обе части полученного
равенства справа на $\ph(e_G)^{-}$:
$$
\ph(e_G)\cdot \ph(e_G)^{-1} = \ph(e_G)\cdot \ph(e_G)\cdot
\ph(e_G)^{-1} = \ph(e_G).
$$
С другой стороны, левая часть очевидным образом равна $e_H$.
Поэтому $e_H = \ph(e_G)$.

Пусть теперь $x\in G$. Тогда $e_H = \ph(e_G) = \ph(x\cdot x^{-1}) =
\ph(x)\cdot \ph(x^{-1})$. 
Домножая обе части на $\ph(x)^{-1}$ слева, видим, что
$\ph(x)^{-1} = \ph(x^{-1})$.
\end{proof}

\begin{examples}
\begin{enumerate}
\item Пусть $G$, $H$~--- произвольные группы. Отображение
  $\const_e\colon G\to H$, $g\mapsto e$, переводящее все элементы
  группы $G$ в нейтральный элемент группы $H$, является гомоморфизмом
  групп. Такой гомоморфизм называется
  \dfn{тривиальным}\index{гомоморфизм!тривиальный}.
  Тождественное отображение $\id_G\colon G\to G$ также является
  гомоморфизмом групп по тривиальным причинам.
\item Пусть $G = (\mb R,+)$~--- аддитивная группа поля $\mb R$, и $H =
  \mb R^*$~--- мультипликативная группа поля $\mb R$. Определим
  отображение $\exp\colon (\mb R,+)\to \mb R^*$ посредством формулы
  $\exp(x) = e^x$, где $e$~--- основание натуральных логарифмов. Это
  гомоморфизм групп, поскольку $e^{x+y} = e^x\cdot e^y$ для всех
  вещественных $x,y$.
\item Пусть теперь $G = (\mb R_{>0},\cdot)$~--- группа положительных
  вещественных чисел с операцией умножения, $H = (\mb R,+)$~---
  аддитивная группа поля $\mb R$. Рассмотрим отображение логарифма
  $\ln\colon (\mb R_{>0},\cdot)\to (\mb R,+)$. Это гомоморфизм групп,
  поскольку $\ln(xy) = \ln(x) + \ln(y)$ для всех вещественных
  $x,y>0$.
\item Пусть $G = S_n$, $H=\{\pm 1\} = \mb Z^*$~--- группа обратимых
  элементов кольца целых чисел. Отображение знака
  $\sgn\colon S_n\to\{\pm 1\}$ является гомоморфизмом групп
  (теорема~\ref{thm:permutation_sign_product}).
\item Пусть $G = H = \mb Z$~--- аддитивная группа целых чисел, и
  $m\in\mb Z$. Определим отображение $\ph\colon\mb Z\to\mb Z$
  умножения на $m$ формулой $\ph(x) = mx$ для всех целых $x$. Нетрудно
  видеть, что $\ph$ является гомоморфизмом групп: $m(x+y) = mx +
  my$. Более общо, если $R$~--- произвольное кольцо, и $m\in R$, то
  отображение $\ph\colon R\to R$, $x\mapsto mx$ является гомоморфизмом
  аддитивной группы $R$ в себя по причине дистрибутивности.
\item Пусть $G = \GL(n,k)$~--- группа обратимых матриц размера
  $n\times n$ над некоторым полем $k$, а $H=k^*$~--- мультипликативная
  группа этого поля. Определитель является гомоморфизмом
  $\det\colon\GL(n,k)\mapsto k^*$, поскольку $\det(xy) =
  \det(x)\det(y)$ для всех $x,y\in\GL(n,k)$
  (теорема~\ref{thm:determinant_product}).
\end{enumerate}
\end{examples}

\begin{definition}
Пусть $\ph\colon G\to H$~--- гомоморфизм групп. \dfn{Ядром}
гомоморфизма $\ph$ называется множество $\Ker(\ph)=\{x\in G\mid
\ph(x) = e_H\}$ (полный прообраз единицы). \dfn{Образом} гомоморфизма
$\ph$ называется его теоретико-множественный образ: $\Img(\ph) =
\{y\in H\mid y = \ph(x)\text{ для некоторого }x\in G\}$.
\end{definition}

\begin{proposition}\label{prop:kernel_and_image}
Образ гомоморфизма $\ph\colon G\to H$ является подгруппой в $H$, а его
ядро~--- {\it нормальной} подгруппой в $G$:
$\Img(\ph)\leq H$, $\Ker(\ph)\trleq G$.
\end{proposition}
\begin{proof}
Пусть $h,h'\in\Img(\ph)$. Это означает, что найдутся $g,g'\in G$ такие,
что $\ph(g) = h$ и $\ph(g') = h'$. Тогда $\ph(gg') = \ph(g)\ph(g') =
hh'$,
откуда следует, что и $hh'\in\Img(\ph)$. Кроме того,
$\ph(g^{-1}) = \ph(g)^{-1} = h^{-1}$, откуда $h^{-1}\in\Img(\ph)$.

Пусть теперь $g,g'\in\Ker(\ph)$. Это означает, что $\ph(g) = e$ и $\ph(g') =
e$. Тогда $\ph(gg') = \ph(g)\ph(g') = e\cdot e = e$, поэтому
$gg'\in\Ker(\ph)$. Кроме того, $\ph(g^{-1}) = \ph(g)^{-1} = e^{-1} = e$,
поэтому и $g^{-1}\in\Ker(\ph)$.

Наконец, если $x\in\Ker(\ph)$, то $\ph(gxg^{-1}) =
\ph(g)\ph(x)\ph(g^{-1}) = \ph(g)\ph(g^{-1}) = \ph(gg^{-1}) = e$, то
есть, $gxg^{-1}$ тоже лежит в $\Ker(\ph)$. Мы показали, что
$g\Ker(\ph)g^{-1}\subseteq\Ker(\ph)$ для любого $g\in G$; по
лемме~\ref{lem:normal_subgroup} этого достаточно для доказательства
нормальности $\Ker(\ph)\trleq G$.
\end{proof}

\begin{remark}
Сравните с предложениями~\ref{prop:kernel-is-subspace}
и~\ref{prop:image-is-subspace}. Здесь нужно быть
аккуратнее: операция в группе, в отличие от сложения в векторном
пространстве, не обязана быть коммутативной. Тем не менее,
доказательство переносится дословно.
\end{remark}

\begin{remark}
Пусть $\ph\colon G\to H$~--- гомоморфизм групп.
Образ $\Img(\ph)$ измеряет отклонение гомоморфизма от сюръективности:
$\ph$ сюръективно тогда и только тогда, когда $\Img(\ph) = H$.
Аналогично, следующая лемма показывает, что ядро $\Ker(\ph)$ измеряет
отклонение $\ph$ от инъективности.
\end{remark}

\begin{lemma}\label{lem:injective_homo}
Пусть $\ph\colon G\to H$~--- гомоморфизм групп. Он инъективен тогда и
только тогда, когда $\Ker(\ph) = \{e\}$.
\end{lemma}
\begin{proof}
Если $\ph$ инъективен, то есть только один элемент $g\in G$ такой, что
$\ph(g) =e$, и мы знаем, что $\ph(e)=e$.
Обратно, если $\Ker(\ph)=\{e\}$ и $g,g'\in G$ таковы, что
$\ph(g)=\ph(g')$, то $\ph(g^{-1}g') = \ph(g)^{-1}\ph(g') = e$, поэтому
$g^{-1}g'\in\Ker(\ph)=\{e\}$, откуда $g = g'$.
\end{proof}

\begin{definition}
Пусть $G, H$~--- группы. Отображение $f\colon G\to H$ называется
\dfn{изоморфизмом групп}, если $f$~--- гомоморфизм групп, и существует
гомоморфизм групп $f'\colon H\to G$ такой, что $f'\circ f = \id_G$ и
$f\circ f' = \id_H$.
\end{definition}

\begin{lemma}\label{lem:bijective_group_homo}
Гомоморфизм групп $f\colon G\to H$ является изоморфизмом тогда и
только тогда, когда $f$ биективен.
\end{lemma}
\begin{proof}
Если $f$ изоморфизм, то у него имеется обратное отображение $f'$, и
поэтому $f$ биективен. Обратно, если $f\colon G\to H$~-- гомоморфизм,
являющийся биекцией, рассмотрим обратное отображение
$f^{-1}\colon H\to G$. Покажем, что это тоже гомоморфизм групп. Нам
нужно проверить, что для любых $h,h'\in H$ выполнено $f^{-1}(h)\cdot
f^{-1}(h') = f^{-1}(hh')$.
Обозначим $f^{-1}(h) = g$, $f^{-1}(h') = g'$; тогда по предположению
$f(gg') = f(g)f(g') = hh'$, откуда $gg'= f^{-1}(hh')$, что и
требовалось.
\end{proof}


\subsection{Фактор-группы}

\literature{[F], гл.~X, \S~1, п. 5, \S~2, \S~3, п. 2; [K3],
гл. 1, \S~4, пп. 1--2; [vdW], гл. 2, \S\S~8, 10; [Bog], гл. 1, \S~2.}

Пусть $G$~--- группа, и $H\trleq G$~--- ее нормальная
подгруппа. Рассмотрим множество $G/H$ правых классов смежности $G$ по
$H$ и введем на нем бинарную операцию: для $gH, g'H\in G/H$ положим
$(gH)\cdot (g'H) = (gg')H$.

\begin{theorem}
Эта операция корректно определена и превращает фактор-множество $G/H$
в группу. Каноническая проекция $G\to G/H$ на фактор-множество
является гомоморфизмом групп.
\end{theorem}
\begin{proof}
Корректная определенность означает, что если мы рассмотрим других
представителей $\widetilde{g}\in gH$ и $\widetilde{g'}\in g'H$, то
результат их перемножения будет тот же:
$(\widetilde{g}\widetilde{g'})H = (gg')H$. Действительно,
запишем $\widetilde{g} = gh$, $\widetilde{g'} = g'h'$; тогда
$\widetilde{g}\widetilde{g'} = ghg'h' = g(hg')h'$. По определению
нормальности элемент $hg'$ можно записать в виде $g'h''$ для
некоторого $h''\in H$; поэтому $\widetilde{g}\widetilde{g'} =
gg'h''h'\in gg'H$. Это и означает, что $\widetilde{g}\widetilde{g'}$
лежит в том же классе, что $gg'$.

Теперь несложно проверить ассоциативность: $(gH\cdot g'H)\cdot
g''H = (gg')H\cdot g''H = (gg')g''H = g(g'g'')H = gH\cdot (g'g'')H =
gH\cdot (g'H\cdot g''H)$. Нейтральным элементом для $G/H$ служит
смежный класс $eH$, поскольку $eH\cdot gH = (eg)H = gH = (ge)H =
gH\cdot eH$. Наконец, у каждого класса $gH$ имеется обратный класс
$g^{-1}H$: $gH\cdot g^{-1}H = eH = g^{-1}H\cdot gH$.

Наконец, утверждение о том, что каноническая проекция $\pi\colon G\to
G/H$ является гомоморфизмом, напрямую следует из определения операции
в $G/H$. Действительно, $\pi(x)\pi(y) = xH\cdot yH$, в то время как
$\pi(xy) = (xy)H$.
\end{proof}

\begin{examples}\label{examples:quotient-groups}
\begin{enumerate}
\item $G/G\isom\{e\}$. Действительно, имеется только один класс
  смежности $G$ по $G$.
\item $G/\{e\}\isom G$: все классы смежности $G$ по подгруппе $\{e\}$
  одноэлементны и поэтому отождествляются с элементами
  $G$. Формула для операции в фактор-группе превращается в
  $g\{e\}\cdot g'\{e\} = gg'\{e\}$, что после отождествления означает,
  что $g\cdot g'$ полагается равным $gg'$; поэтому операция в
  $G/\{e\}$ та же, что была в $G$.
\item Мы уже встречали группу $\mb Z/m\mb Z$: это аддитивная группа
  кольца вычетов по модулю $m$.
\item\label{item:angles-as-quotient-group}
  Рассмотрим аддитивную группу поля вещественных чисел $\mbR$
  и подгруппу $2\pi\mbZ = \{2\pi n\mid n\in\mbZ\}$ в ней.
  Фактор-группу $\mbR/2\pi\mbZ$ естественно представлять как множество
  вещественных чисел <<с точностью до целых кратных $2\pi$>>. Например,
  в этой группе есть элемент $3\pi/2$ (точнее, образ элемента
  $3\pi/2\in\mbR$ относительно канонической проекции) и элемент
  $\pi$. Их сумма равна $3\pi/2 + \pi = 5\pi/2 = \pi/2\in\mb R/2\pi\mbZ$,
  поскольку сложение происходит <<по модулю $2\pi$>>.
  Нетрудно понять, что эта группа изоморфна группе $\mb T$ комплексных
  чисел модуля $1$
  (см. пример~\ref{examples:group}~(\ref{item:group_of_angles}))~---
  изоморфизм устанавливается взятием аргумента.
  Поэтому группа $\mbR/2\pi\mbZ$, как и группа $\mb T$, часто
  называется \dfn{группой углов}.\index{группа!углов}
\end{enumerate}
\end{examples}

Теперь мы можем доказать аналог теоремы о
гомоморфизме~\ref{thm_homomorphism}.

\begin{theorem}[Теорема о гомоморфизме]\label{thm:homomorphism_groups}
Пусть $G, H$~--- группы, $\ph\colon G\to H$~--- гомоморфизм
групп. Тогда $G/\Ker(\ph)\isom\Img(\ph)$.
\end{theorem}

\begin{proof}
Определим отображение $\widetilde\ph\colon G/\Ker(\ph)\to\Img(\ph)$
правилом $\widetilde\ph(g\Ker(\ph)) = \ph(g)$. Заметим, прежде всего,
что $\ph(g)$ действительно лежит в $\Img(\ph)$. Далее, этот
гомоморфизм корректно определен: если $g\Ker(\ph) = g'\Ker(\ph)$, то
$g = g'x$ для некоторого $x\in\Ker(\ph)$, поэтому
$\ph(g) = \ph(g'x) = \ph(g')\ph(x) = \ph(g')e = \ph(g')$.

Проверим, что $\widetilde\ph$~--- изоморфизм групп. Для этого по
лемме~\ref{lem:bijective_group_homo} достаточно проверить, что
$\widetilde\ph$~--- биективный гомоморфизм групп. Пусть
$g\Ker(\ph), g'\Ker(\ph)\in G/\Ker(\ph)$.
Тогда $\widetilde\ph(g\Ker(\ph))\widetilde\ph(g'\Ker(\ph)) =
\ph(g)\ph(g')$ и $\widetilde\ph(g\Ker(\ph)\cdot g'\Ker(\ph)) =
\widetilde\ph((gg')\Ker(\ph)) = \ph(gg')$. Получили одно и то же
(поскольку $\ph$~--- гомоморфизм групп).

Для доказательства биективности проверим инъективность и
сюръективность. Инъективность: по лемме~\ref{lem:injective_homo}
достаточно показать, что ядро $\widetilde\ph$ тривиально. Если
$g\Ker(\ph)$ лежит в этом ядре, то $\widetilde\ph(g\Ker(\ph)) = \ph(g)
= e$, поэтому $g\in\Ker(\ph)$ и $g\Ker(\ph) = e\Ker(\ph)$, что и
требовалось. Сюръективность: если $h\in\Img(\ph)$, то найдется $g\in
G$ такой, что $\ph(g) = h$. Но тогда $\widetilde\ph(g\Ker(\ph)) =
\ph(g) = h$.
\end{proof}

\subsection{Циклические группы}

\literature{[F], гл.~X, \S~1, пп. 6--7; [K1], гл. 4, \S~2, п. 2; [K3],
гл. 1, \S~2, п. 2; [vdW], гл. 2, \S~7.}

Пусть $G$~--- произвольная группа, $g\in G$. Определим отображение
$\pow_g\colon\mb Z\to G$ следующим образом: целое число $n$ отправим в
$g^n\in
G$. Иными словами, для натурального $n$ положим
$g^n = \underbrace{g\cdot\dots\cdot g}_n$ и
$g^{-n} = \underbrace{g^{-1}\cdot\dots\cdot g^{-1}}_n$. Легко видеть,
что при этом $g^{m+n} = g^m\cdot g^n$ для всех $m,n\in\mb Z$ поэтому
отображение $\pow_g$ является гомоморфизмом групп.
Его образ по предложению~\ref{prop:kernel_and_image} является
подгруппой в $G$.

\begin{lemma}\label{lem:image_power_g}
Образ отображения $\pow_g$ совпадает с $\la g\ra$ (подгруппой,
порожденная $g$).
\end{lemma}
\begin{proof}
Прежде всего, $\Img(\pow_g)$ содержит $g$, поэтому и
$\la g\ra\subseteq\Img(\pow_g)$. С другой стороны,
любой элемент $\Img(\pow_g)$ имеет вид $g^n$ для некоторого $n$, и
содержится в $\la g\ra$, поскольку $\la g\ra$~--- подгруппа в $G$.
\end{proof}

\begin{definition}
Группа $G$ называется \dfn{циклической}\index{группа!циклическая},
если она порождается одним элементом, то есть, найдется элемент
$g\in G$ такой, что $G=\la g\ra$.
\end{definition}

Наша ближайшая задача~--- описать все циклические группы.

\begin{theorem}[Классификация циклических групп]\label{thm:cyclic_groups}
Любая циклическая группа изоморфна $\mb Z/m\mb Z$ для некоторого
натурального $m$. В случае $m=0$ получаем бесконечную циклическую
группу $\mb Z$, в остальных случаях получаем циклическую группу из $m$ элементов.
\end{theorem}
\begin{proof}
Пусть $G$~--- циклическая группа, порожденная элементом $g\in
G$. Рассмотрим отображение $\pow_g\colon\mb Z\to G$. По
лемме~\ref{lem:image_power_g} его образ совпадает с $\la g\ra = G$. По
теореме о гомоморфизме~\ref{thm:homomorphism_groups} имеем
$\mb Z/\Ker(\pow_g)\isom G$.
По теореме~\ref{thm:subgroups_of_z} $\Ker(\pow_g)$, будучи подгруппой
в $\mb Z$, имеет вид $m\mb Z$ для некоторого натурального $m$, что и
требовалось доказать.
\end{proof}

\begin{corollary}
Пусть $G$~--- произвольная группа, $g\in G$. Множество $\{g^n\mid
n\in\mb Z\}$ является подгруппой в $G$, изоморфной группе $\mb Z/m\mb
Z$ для некоторого $m\in\mb N$.
\end{corollary}
\begin{proof}
Это множество~--- циклическая подгруппа $\la g\ra$; осталось применить
к ней теорему~\ref{thm:cyclic_groups}.
\end{proof}

\begin{definition}
Если группа $\{g^n\mid n\in\mb Z\}$ изоморфна $\mb Z/m\mb Z$ и $m>0$,
говорят, что элемент $g$ имеет \dfn{порядок}\index{порядок!элемента в
  группе} $m$. Если же эта группа изоморфна $\mb Z$, то говорят, что
$g$ имеет \dfn{бесконечный порядок}. Таким образом,
порядок элемента $g$ равен числу элементов в циклической подгруппе
$\la g\ra$, порожденной $g$.
Обозначение для порядка:
$\ord_G(g) = m\text{ или }\infty$.
\end{definition}

Иными словами, порядок элемента $g\in G$~--- это наименьшее
натуральное число $m$ такое, что $g^m=1$. Действительно, при
гомоморфизме $\pow_g\colon\mb Z\to G$ в единицу переходят в точности
элементы из подгруппы $m\mb Z$.

\begin{remark}\label{rem:order_of_neutral_element}
Заметим, что порядок нейтрального элемента равен $1$, и это
единственный элемент порядка $1$ в любой группе.
\end{remark}


\subsection{Теорема Лагранжа}

\literature{[F], гл.~X, \S~1, пп. 5, 7; [K3], гл. 1, \S~2, п. 1;
  [Bog], гл. 1, \S~2.}

\begin{definition}
Пусть $G$~--- группа, $H\leq G$. Количество правых смежных классов $G$
по $H$ называется \dfn{индексом}\index{индекс подгруппы} подгруппы $H$
и обозначается через $|G:H|$.
\end{definition}

Покажем, что в этом определении можно заменить правые смежные классы
на левые смежные классы:

\begin{lemma}
Пусть $G$~--- группа, $H\leq G$. Тогда множества левых смежных классов
$G$ по $H$ и правых смежных классов $G$ по $H$ равномощны.
\end{lemma}
\begin{proof}
Пусть $\{a_iH\}_{i\in I}$~--- множество всех правых смежных классов
(иными словами, мы выбрали в каждом правом смежном классе по
представителю и занумеровали их элементами некоторого множества $I$,
возможно, бесконечного). 
По предложению~\ref{prop:group_cosets} каждый элемент группы $G$
содержится ровно в одном множестве вида $a_iH$. Покажем, что
набор $\{Ha_i^{-1}\}_{i\in I}$ состоит из всех левых смежных классов,
взятых ровно по одному разу (то есть, что $a_i^{-1}$~--- представители
всех левых смежных классов $G$ по $H$).

Действительно, пусть $g\in G$. Тогда $g\in Ha_i^{-1}$ равносильно тому, что
$g=ha_i^{-1}$ для некоторого $H$, откуда $g^{-1} = (ha_i^{-1})^{-1} =
a_ih^{-1}\in a_iH$. Но это равенство выполнено ровно для одного
индекса $i\in I$, поэтому $g$ лежит ровно в одном множестве вида
$Ha_i^{-1}$, что и требовалось доказать.
\end{proof}

\begin{remark}
По определению фактор-множество $G/H$ состоит из правых смежных
классов $G$ по $H$, так что $|G:H| = |G/H|$.
\end{remark}

\begin{theorem}[Теорема Лагранжа]
Пусть $G$~--- конечная группа, $H\leq G$. Тогда
$|G| = |H|\cdot |G:H|$.
\end{theorem}
\begin{proof}
Докажем, что во всех правых смежных классах $G$ по $H$ поровну
элементов. Заметим, что для каждого $g\in G$ отображение $H\to gH$,
$h\mapsto gh$, задает биекцию между $H$ и $gH$. Действительно, если
$gh=gh'$, то $h=h'$, и в силу определения смежного класса это
отображение сюръективно. Поэтому в каждом смежном классе столько же
элементов, сколько в подгруппе $H$. Таким образом, элементы $G$
разбиваются на $|G:H|$ смежных классов, в каждом по $|H|$
элементов. Отсюда сразу следует требуемое равенство.
\end{proof}
\begin{corollary}\label{cor:order_divides}
Порядок конечной группы $G$ делится на порядок любой ее подгруппы. В
частности, порядок конечной группы $G$ делится на порядок любого ее
элемента.
\end{corollary}
\begin{proof}
Первое утверждение очевидно; второе следует из первого, если
рассмотреть подгруппу $\la g\ra$, порядок которой (по определению)
равен порядку $g$.
\end{proof}

\begin{corollary}\label{cor:power_order}
Пусть $G$~--- конечная группа. Тогда $g^{|G|} = e$ для любого $g\in G$.
\end{corollary}

В качестве примера приложения теоремы Лагранжа выведем из нее теорему
Эйлера~\ref{thm:euler} (и, как следствие, малую теорему
Ферма~\ref{cor_fermat}).

\begin{theorem}
Пусть $m$~--- натуральное число, $a\in\mb Z$ и $a\perp m$. Тогда
$a^{\ph(m)}\equiv 1\pmod m$.
\end{theorem}
\begin{proof}
Рассмотрим кольцо $\mb Z/m\mb Z$. Множество $(\mb Z/m\mb Z)^*$ его
обратимых элементов образует группу по умножению
(пример~\ref{examples:group} (\ref{item:group_of_units})). Порядок этой
группы равен $\ph(m)$ (предложение~\ref{prop_phi_alt_def}).
Класс $\overline{a}$ элемента $a$ в $\mb Z/m\mb Z$ обратим, поскольку
$a\perp m$ (предложение~\ref{prop_invertibility_criteria}).
Применение следствия~\ref{cor:power_order} дает
$\overline{a}^{\ph(m)}=\overline{1}$, что в переводе на язык целых
чисел и дает нужное равенство.
\end{proof}

Еще одно приложение теоремы Лагранжа~--- описание всех групп простого
порядка.

\begin{theorem}\label{thm:groups_of_prime_order}
Пусть $G$~--- конечная группа порядка $p$, где $p$~--- простое число.
Тогда $G$ изоморфна циклической группе $\mb Z/p\mb Z$.
\end{theorem}
\begin{proof}
По теореме Лагранжа
порядок любого элемента группы $G$ должен быть делителем $p$, и в силу
простоты $p$ он равен либо $1$ либо $p$. По
замечанию~\ref{rem:order_of_neutral_element} в
$G$ лишь один элемент имеет порядок $1$; поэтому найдется элемент
$g\in G$ порядка $p$. Но тогда подгруппа $\la g\ra$ состоит из $p$
элементов и, стало быть, совпадает с $G$. Значит, $G$ циклическая,
порождена элементом $g$ и (по теореме~\ref{thm:cyclic_groups})
изоморфна $\mb Z/p\mb Z$.
\end{proof}

\subsection{Прямое произведение}

\literature{[F], гл.~X, \S~4, пп. 1--2, [K3], гл. 1, \S~4, п. 4.}

Пусть $G,H$~--- две группы.
Рассмотрим декартово произведение множеств $G\times H$ и введем на нем
операцию: положим $(g,h)\cdot (g',h') = (gg',hh')$ для $g,g'\in G$,
$h,h'\in H$.
Нетрудно видеть, что $G\times H$ с такой операцией является группой:
ассоциативность выполняется, поскольку она выполняется в группах $G$ и
$H$, нейтральным элементом служит пара $(e,e)$, обратным элементом к
паре $(g,h)$ является элемент $(g^{-1},h^{-1})$.

\begin{definition}
Множество $G\times H$ с такой операцией называется
\dfn{прямым произведением}\index{прямое произведение!групп} групп $G$
и $H$.
\end{definition}

\begin{proposition}\label{prop:direct_product_properties}
Пусть $G,H$~--- группы.
Рассмотрим отображения
\begin{align*}
i_1\colon G\to G\times H,&\;\; g\mapsto (g,e),\\
i_2\colon H\to G\times H,&\;\; h\mapsto (e,h),\\
\pi_1\colon G\times H\to G,&\;\; (g,h)\mapsto g,\\
\pi_2\colon G\times H\to H,&\;\; (g,h)\mapsto h.
\end{align*}
\begin{enumerate}
\item $i_1,i_2$~--- инъективные, а $\pi_1,\pi_2$~--- сюръективные
  гомоморфизмы групп;
\item\label{item:direct_product_2}
  $\Img(i_1)=\Ker(\pi_2)=G\times\{e\}$,
  $\Img(i_2)=\Ker(\pi_1)=\{e\}\times H$~--- нормальные подгруппы в
  $G\times H$;
\item $\pi_1\circ i_1 = \id_G$, $\pi_2\circ i_2 = \id_H$;
  $\pi_1\circ i_2 = e$, $\pi_2\circ i_1 = e$;
\end{enumerate}
\end{proposition}
\begin{proof}
\begin{enumerate}
\item Очевидно.
\item $\Img(i_1)$ состоит в точности из элементов вида $(g,e)$, а
  $\Ker(\pi_2)$ состоит из элементов $(g,h)$ таких, что $h=e$; и то, и
  другое совпадает с $G\times\{e\} = \{(g,e)\in G\times H\mid g\in
  G\}$. Нормальность следует из
  предложения~\ref{prop:kernel_and_image}. Оставшееся аналогично.
\item $\pi_1(i_1(g)) = \pi_1((g,e)) = g$, $\pi_2(i_1(g)) =
  \pi_2((g,e)) = e$. Оставшееся аналогично.
\end{enumerate}
\end{proof}

Таким образом, отображения $i_1$, $i_2$ устанавливают изоморфизмы
$G\isom G\times\{e\}$ и $H\isom \{e\}\times H$ между группами $G,H$ и
подгруппами в $G\times H$. Естественно поинтересоваться, когда верно
обратное: когда в данной группе $F$ можно найти две подгруппы $G$,
$H$ такие, что $F$ изоморфно прямому произведению $G\times H$, и
подгруппы $G$, $H$ получаются посредством вложений $i_1$, $i_2$ для
этого прямого произведения? Ответ дает следующая теорема.

\begin{theorem}\label{thm:direct_product}
Пусть $F$~--- группа. Пусть $G\leq F$, $H\leq F$~--- две подгруппы в
$F$. Обозначим через $j_1\colon G\to F$, $j_2\colon H\to F$
соответствующие вложения.
Предположим, что выполнены следующие условия:
\begin{enumerate}
\item\label{item:intersection_is_trivial} $G\cap H = \{e\}$
  (пересечение этих подгрупп тривиально);
\item\label{item:generate_all} $GH=F$ (любой элемент $x$ группы $F$
  можно записать в виде $x = gh$ для некоторых $g\in G$, $h\in H$);
\item\label{item:they_commute} $gh=hg$ для всех $g\in G$, $h\in H$
  (подгруппы $G$ и $H$ коммутируют).
\end{enumerate}
Тогда группа $F$ изоморфна прямому произведению $G$ и $H$; более
того, существует такой изоморфизм $\ph\colon F\to G\times H$,
что композиция
$$
\pi_1\circ\ph\circ j_1\colon G\to F\to G\times H\to G
$$
является тождественным отображением на $G$, а композиция
$$
\pi_2\circ\ph\circ j_2\colon H\to F\to G\times H\to H
$$
является тождественным отображением на $H$.
\end{theorem}
\begin{proof}
Построим изоморфизм $\ph$. Возьмем $x\in F$ и запишем его (пользуясь
свойством~\ref{item:generate_all}) в виде $x = gh$, где $g\in G$ и
$h\in
H$. Заметим, что такое представление единственно: если $x = g'h'$ для
$g'\in G$, $h'\in H$, то $gh=g'h'$, откуда 
$g'^{-1}g = h'h^{-1}$; в левой части стоит элемент $G$, а в правой~---
элемент $H$, значит (по свойству~\ref{item:intersection_is_trivial})
$g'^{-1}g = e = h'h^{-1}$, откуда $g=g'$ и $h=h'$.
Поэтому мы можем положить $\ph(x) = (g,h)$.

Проверим, что $\ph$~--- гомоморфизм групп. Возьмем $y\in F$ и запишем
его в виде $y = g'h'$, где $g',h'\in H$.
Тогда $xy = (gh)(g'h') = g(hg')h' = (gg')(hh')$ (по
свойству~\ref{item:they_commute}). По определению $\ph$ теперь
$\ph(xy) = (gg',hh')$, в то время как $\ph(x) = (g,h)$, $\ph(y) =
(g',h')$, и, стало быть, $\ph(x)\ph(y) = (g,h)(g',h') = (gg', hh')$.

Для доказательства инъективности $\ph$ достаточно проверить
тривиальность его ядра (лемма~\ref{lem:injective_homo}). Но если
$\ph(x) = (e,e)$, то $x = ee = e$. Для всех пар $(g,h)\in
G\times H$ найдется $x=gh\in F$ такой, что $\ph(x)=(g,h)$, поэтому
$\ph$ сюръективен.
Наконец, $\pi_1(\ph(j_1(g))) = \pi_1(\ph(g)) = \pi_1((g,e)) = g$ и
$\pi_2(\ph(j_2(h))) = \pi_2(\ph(h)) = \pi_2((e,h)) = h$.
\end{proof}

\subsection{Симметрическая группа}

\literature{[F], гл.~X, \S~5, п. 4; [K1], гл. 1, \S~8, п. 2, гл. 4,
  \S~2, п. 3; [Bog], гл. 1, \S~4.}

Сейчас мы вернемся к изучению группы $S_n$.

\begin{definition}
Перестановка $\pi\in S_n$ называется
\dfn{циклом длины $k$}\index{цикл}, если для
некоторых различных $i_1,\dots,i_k\in\{1,\dots,n\}$ выполнено
$\pi(i_1) = i_2$, $\pi(i_2) = i_3$, \dots, $\pi(i_{k-1}) = i_k$,
$\pi(i_k) = i_1$, и для всех
$j\in\{1,\dots,n\}\setminus\{i_1,\dots,i_k\}$ выполнено $\pi(j)=j$.
Такой цикл мы будем обозначать так:
$(i_1\;\;i_2\;\;\dots i_k)$.
При этом множество $\{i_1,\dots,i_k\}\subseteq\{1,\dots,n\}$
называется \dfn{носителем}\index{носитель цикла} цикла $\pi$.
Два цикла $\pi,\rho\in S_n$ называются
\dfn{независимыми}\index{независимые циклы}, если их носители не
пересекаются. Заметим, что циклы длины $1$ не очень полезно
рассматривать: это тождественная перестановка.
\end{definition}

\begin{remark}\label{rem:different_notations_cycle}
Заметим, что цикл длины $k$ можно записать $k$ различными способами:
$(i_1\;\;i_2\;\;\dots\;\;i_{k-1}\;\;i_k) = 
(i_2\;\;i_3\;\;\dots\;\;i_k\;\;i_1) = \dots =
(i_k\;\;i_1\;\;\dots\;\;i_{k-2}\;\;i_{k-1})$.
\end{remark}

\begin{lemma}
Независимые циклы коммутируют: если $\pi,\rho\in S_n$~--- независимые
циклы, то $\pi\rho = \rho\pi$.
\end{lemma}
\begin{proof}
Непосредственное вычисление.
\end{proof}

\begin{definition}
Пусть $\pi\in S_n$. Множество $\Fix(\pi) = \{i\in\{1,\dots,n\}\mid
\pi(i)=i\}$ называется \dfn{множеством неподвижных
  точек} перестановки $\pi$, а его
элементы~--- \dfn{неподвижными точками}\index{неподвижные точки
  перестановки} $\pi$.
\end{definition}

\begin{theorem}
Любую перестановку $\pi\in S_n$ можно представить в виде произведения
независимых циклов, носители которых не пересекаются с $\Fix(\pi)$.
\end{theorem}
\begin{proof}
Будем вести индукцию по числу $i\in\{1,\dots,n\}$ таких, что
$\pi(i)\neq i$, то есть, по $n-|\Fix(\pi)|$.
Если это число равно $0$, то перестановка $\pi$
тождественна и, таким образом, есть произведение пустого множества
циклов. Это база индукции. Докажем переход.
Пусть теперь множество $I = \{i\in\{1,\dots,n\}\mid \pi(i)\neq i\}$
непусто; например, $i_1\in I$. Рассмотрим последовательность
$i_1,\pi(i_1),\pi^2(i_1),\dots$. По предположению
$i_1\neq\pi(i_1)$. Рассмотрим первый элемент этой последовательности,
совпадающий с каким-то из ранее встретившихся: такой найдется,
поскольку все элементы этой последовательности лежат в конечном
множестве $\{1,\dots,n\}$. Пусть это $\pi^k(i_1) =
\pi^l(i_1)$ при $k>l$. Если $l>0$, ты применяя к этому равенству
$\pi^{-1}$, получаем $\pi^{k-1}(i_1) = \pi^{l-1}(i_1)$, что
противоречит предположению о минимальности $k$. Значит,
$l=0$ и $\pi^k(i_1) = i_1$. Кроме того, опять же в силу минимальности
$k$, все элементы $i_1,\pi(i_1),\pi^2(i_1),\dots,\pi^{k-1}(i_1)$
различны. Обозначим
$i_2=\pi(i_1),i_3=\pi^2(i_1),\dots,i_k=\pi^{k-1}(i_1)$ и рассмотрим
цикл $\sigma=(i_1\;\;i_2\;\;\dots\;\;i_k)$. Мы знаем, что
$\pi(i_1)=i_2$, $\pi(i_2)=i_3$, \dots, $\pi(i_{k-1})=i_k$ и
$\pi(i_k) = i_1$, поэтому произведение
$\pi' = \sigma^{-1}\circ\pi$ обладает следующим свойством:
$\pi'(i_1) = i_1$, $\pi'(i_2) = i_2$, \dots, $\pi'(i_k) = i_k$,
и $\pi'(j)=\pi(j)$ для всех
$j\in\{1,\dots,n\}\setminus\{i_1,\dots,i_k\}$.

Это значит, что к $\pi'$ можно применить предположение индукции:
действительно, $\Fix(\pi') = \Fix(\pi)\cup\{i_1,\dots,i_k\}$, поэтому
мощность множества $\{i\in\{1,\dots,n\}\mid \pi'(i)\neq i\}$ на $k$
меньше, чем мощность аналогичного множества для $\pi$.
По предположению индукции $\pi'$ можно записать в виде произведения
независимых циклов, носители которых не пересекаются с $\Fix(\pi')$:
$\pi' = \tau_1\dots\tau_s$. После этого остается записать
$\pi = \sigma\pi' = \sigma\tau_1\dots\tau_s$ и заметить, что носитель
цикла $\sigma$~--- это множество $\{i_1,\dots,i_k\}$, не
пересекающееся с $\Fix(\pi) = \Fix(\pi')\setminus\{i_1,\dots,i_k\}$.
\end{proof}

\begin{definition}
Запись элемента $\pi\in S_n$ в виде, указанном в теореме,
называется \dfn{цикленной записью перестановки}\index{цикленная запись
  перестановки} $\pi$.
\end{definition}

\begin{example}
Цикленные записи нетождественных перестановок из $S_3$ выглядят так:
$(1\;\;2)$, $(1\;\;3)$, $(2\;\;3)$, $(1\;\;2\;\;3)$,
$(1\;\;3\;\;2)$. Цикленная запись тождественной перестановки пуста.
В $S_4$ имеются три перестановки, в цикленной записи которых более
одного цикла: $(1\;\;2)(3\;\;4)$, $(1\;\;3)(2\;\;4)$,
$(1\;\;4)(2\;\;3)$.
\end{example}

\begin{remark}
Как мы видели выше (замечание~\ref{rem:different_notations_cycle}),
запись цикла в виде $(i_1\;\;i_2\;\;\dots\;\;i_k)$ не вполне
однозначна: на первое место можно поставить любой элемент из
$i_1,\dots,i_k$. Кроме того, в произведении нескольких независимых
циклов их можно переставлять местами произвольным образом (независимые
циклы коммутируют). Несложно понять, что в остальном циклическая
запись перестановки единственна. Действительно, каждое число от $1$ до
$n$ либо не встречается ни в одном из циклов (и тогда это неподвижная
точка), либо встречается ровно в одном цикле (поскольку циклы
независимы), и тогда его образ однозначно определен. Часто для
удобства в каждом цикле
$(i_1\;\;i_2\;\;\dots\;\;i_k)$ на первое место ставят минимальный
элемент из $i_1,\dots,i_k$, а все циклы в цикленной записи располагают
в порядке возрастания первых элементов этих циклов. 
\end{remark}

Цикленная запись полезна, среди прочего, для визуализации сопряжения
перестановки.

\begin{lemma}\label{lem:cycle_conjugation}
Пусть $\pi\in S_n$, $i_1,\dots,i_k$~--- различные элементы
$\{1,\dots,n\}$. Тогда
$$
{}^\pi(i_1\;\;i_2\;\;\dots\;\;i_k) =
(\pi(i_1)\;\;\pi(i_2)\;\;\dots\;\;\pi(i_k)).
$$
Таким образом, сопряженный элемент к циклу длины $k$ также является
циклом длины $k$.
\end{lemma}
\begin{proof}
Пусть $\pi'= {}^\pi(i_1\;\;i_2\;\;\dots\;\;i_k)$. Применяя
$\pi'$ к $\pi(i_s)$, получаем
$\pi'(\pi(i_s)) = (\pi\circ(i_1\;\;i_2\;\;\dots\;\;i_k))(i_s)
= \pi(i_{s+1})$ при $s<k$ и $\pi(i_1)$ при $s=k$.
Если же $j\in\{1,\dots,n\}$ не совпадает ни с одним из
$\pi(i_1),\dots,\pi(i_k)$, то $\pi^{-1}(j)$ не совпадает ни с одним из
$i_1,\dots,i_k$, поэтому
$\pi'(j) = (\pi\circ(i_1\;\;i_2\;\;\dots\;\;i_k))(\pi^{-1}(j))
= \pi(\pi^{-1}(j)) = j$.
Значит, элементы $\pi(i_1),\dots,\pi(i_k)$ под действием
$\pi'$ сдвигаются по циклу (в указанном порядке), а остальные остаются
на месте.
\end{proof}

\begin{definition}
Пусть $\pi\in S_n$. Набор длин циклов в цикленной записи
$\pi$ (с учетом кратностей) называется \dfn{цикленным типом}
перестановки $\pi$. Так, к примеру, цикленный тип перестановки
$(1\;\;2\;\;3)$ равен $\{3\}$, а перестановки $(1\;\;2)(3\;\;4)$~---
$\{2,2\}$.
\end{definition}

\begin{theorem}\label{thm:cycles_and_conjugation_classes}
Цикленные типы двух сопряженных перестановок одинаковы. Обратно, если
у двух перестановок цикленные типы совпадают, то они сопряжены.
\end{theorem}

\begin{proof}
Если $\pi,\rho\in S_n$ и $\rho=\rho_1\rho_2\dots\rho_s$~--- разложение
перестановки $\rho$ в произведение независимых циклов,
то ${}^\pi\rho = \pi\rho\pi^{-1} = \pi\rho_1\rho_2\dots\rho_s\pi^{-1}
= \pi\rho_1\pi^{-1}\pi\rho_2\pi^{-1}\dots\pi\rho_s\pi^{-1} =
{}^\pi\rho_1\cdot {}^\pi\rho_2\cdot\dots\cdot {}^\pi\rho_s$. Поскольку
при сопряжении цикла получается цикл той же длины, первая часть
теоремы доказана.

Пусть теперь $\rho=\rho_1\rho_2\dots\rho_s$ и
$\tau=\tau_1\tau_2\dots\tau_t$~--- разложения перестановок из $S_n$ в
произведения независимых циклов с одинаковым цикленным типом. Это
означает, что $s=t$ и после перестановки сомножителей можно считать,
что циклы $\rho_i$ и $\tau_i$ имеют одинаковую длину для всех
$i=1,\dots,s$. Укажем перестановку $\pi\in S_n$ такую, что
$\tau = {}^\pi\rho$. Пусть цикл $\rho_1$ имеет вид
$\rho_1 = (i_1\;\;i_2\;\;\dots\;\;i_k)$, а цикл $\tau_1$ имеет вид
$\tau_1 = (j_1\;\;j_2\;\;\dots\;\;j_k)$.
Положим $\pi(i_1) = j_1$, $\pi(i_2) = j_2$, \dots, $\pi(i_k) = j_k$.
Совершим такую же процедуру с циклами $\rho_2$ и $\tau_2$, \dots,
$\rho_s$ и $\tau_s$. Заметим, что все элементы, входящие в записи
циклов $\rho_1,\rho_2,\dots,\rho_s$ попарно различны, так что
противоречия не возникнет. Кроме того, все элементы, входящие в записи
циклов $\tau_1,\tau_2,\dots,\tau_s$ попарно различны, так что пока что
$\pi$ принимает различные значения, которых столько же, сколько всего
элементов в циклах $\rho_1,\rho_2\dots,\rho_s$.
Для элементов $j\in\{1,\dots,n\}$, которые
не входят ни в один из циклов $\rho_1,\rho_2,\dots,\rho_s$, положим
$\pi(j)$ равным произвольным различным элементам, не входящим ни в
один из циклов $\tau_1,\tau_2,\dots,\tau_s$. Это можно сделать,
поскольку их поровну. Легко видеть, что мы получили биекцию $\pi\in
S_n$ и в силу леммы~\ref{lem:cycle_conjugation} имеем
${}^\pi\rho_i = \tau_i$ для всех $i=1,\dots,n$. Поэтому
и ${}^\pi\rho = \tau$.
\end{proof}

\begin{remark}
Из доказательства теоремы~\ref{thm:cycles_and_conjugation_classes}
видно, что искомая перестановка $\pi$, как правило, далеко не
единственна.
\end{remark}

Следующая теорема показывает, что изучение симметрических групп может
быть важным шагом в изучении всех конечных групп.

\begin{theorem}[Теорема Кэли]
Любая конечная группа $G$ изоморфна некоторой подгруппе группы $S_n$
для некоторого натурального $n$.
\end{theorem}
\begin{proof}
Положим $n = |G|$. Занумеруем элементы группы $G$ числами от $1$ до
$n$: $G = \{g_1,\dots,g_n\}$.
Сопоставим каждому элементу $g\in G$ перестановку $\pi_g\in S_n$
следующим образом: для $i=1,\dots,n$ посмотрим на элемент $gg_i$
в группе $G$. Этот элемент должен иметь некоторый номер; его и возьмем
в качестве $\pi_g(i)$. Таким образом, $gg_i = g_{\pi_g(i)}$ для всех
$i$. Прежде всего, нужно показать, что $\pi_g$ действительно является
перестановкой. Инъективность $\pi_g$ показать легко: если $\pi_g(i) =
\pi_g(j)$, то $gg_i = gg_j$, откуда $g_i = g_j$ и $i=j$. Биективность
теперь следует из того, что $\pi_g$ действует на конечном множестве
$\{1,\dots,n\}$ (принцип Дирихле).

Мы построили по каждому элементу $g\in G$ перестановку $\pi_g\in S_n$;
покажем теперь, что соответствие $\pi\colon g\mapsto \pi_g$ является
гомоморфизмом групп. Необходимо показать,
что $\pi_{gg'} = \pi_g\circ\pi_g'$.
Но для каждого $i=1,\dots,n$ имеем
$(gg')g_i = g_{\pi_{gg'}(i)}$; с другой стороны,
$g(g'g_i) = gg_{\pi_{g'}(i)} = g_{\pi_g(\pi_{g'}(i))}$.
Поэтому $\pi_{gg'}(i) = \pi_g(\pi_{g'}(i))$ для всех $i$, что и
требовалось.

Наконец, гомоморфизм $\pi$ инъективен, поскольку
из $\pi_g = \pi_h$ следует $gg_1 = g_{\pi_g(1)} = g_{\pi_h(1)} = hg_1$
и, после сокращения на $g_1$, $g = h$.
Мы построили инъективный гомоморфизм $\pi\colon G\to S_n$; его образ
$\Img(\pi)$ по теореме о гомоморфизме~\ref{thm:homomorphism_groups}
изоморфен фактору $G$
по ядру гомоморфизма $\pi$, которое тривиально. Поэтому группа
$\Img(\pi)$ изоморфна $G$ и является подгруппой в $S_n$.
\end{proof}

\subsection{Диэдральная группа}

\literature{[K3], гл. 1, \S~4, п. 5.}
\nopagebreak

Рассмотрим на эвклидовой плоскости правильный $n$-угольник с вершинами
$A_1,\dots,A_n$ и центром в начале координат (точке $O$).
Множество всех поворотов плоскости, переводящих этот $n$-угольник в
себя, образует группу (см. пример~\ref{examples:group}
(\ref{item:geometric_groups})).
Нетрудно понять, что это циклическая группа: в качестве образующей
можно взять поворот с центром в $O$ на угол $2\pi/n$ в положительном
направлении (whatever this means). Обозначим этот поворот через $x$.
Любой поворот, переводящий $n$-угольник в себя, должен переводить
вершины в вершины: пусть он переводит $A_1$ в $A_k$.
Тогда $A_2$ переходит в $A_{k+1}$, и так далее (если считать, что
вершины занумерованы в положительном направлении, и номера понимаются
по модулю $n$, то есть, $A_{n+1} = A_1$, $A_{n+2} = A_2$,
\dots). Таким образом, этот поворот совпадает с $x^k$.

Рассмотрим теперь множество {\it всех движений} плоскости, переводящих
наш правильный $n$-угольник в себя. Это тоже группа; обозначим ее
через $D_n$.
Она содержит в качестве подгруппы, порожденной элементом $x$,
циклическую группу порядка $n$.
Кроме того, в ней содержатся некоторые осевые симметрии: их описание
зависит от четности $n$. Для нечетного $n$ ось каждой симметрии
проходит через вершину и середину противоположной ей стороны
(например, через вершину $A_1$ и середину стороны
$A_{\frac{n+1}{2}}A_{\frac{n+3}{2}}$): таких симметрий $n$.
Для четного $n$ имеется $n/2$ симметрий относительно прямых,
соединяющих противоположные вершины (например,
$A_1A_{\frac{n}{2}+1}$), и $n/2$ симметрий относительно прямых,
соединяющих середины противоположных сторон (например, середину
стороны $A_1A_2$ с серединой стороны
$A_{\frac{n}{2}+1}A_{\frac{n}{2}+2}$).
В любом случае, всего осевых симметрий ровно $n$, и можно показать,
что они вместе с $n$ поворотами исчерпывают все элементы группы
$D_n$. Таким образом, $|D_n| = 2n$.

Для подробного изучения группы $D_n$ мы будем пользоваться ее
{\it матричным представлением}. А именно, заметим, что все описанные
повороты и симметрии сохраняют точку $O$. Движение эвклидовой
плоскости, сохраняющее точку $O$, является, среди прочего, линейным
отображением соответствующего двумерного векторного
пространства. Поэтому после выбора ортогонального базиса можно
отождествить элементы группы $D_n$ с их матрицами в этом базисе.
Нетрудно понять, что
$$
x = \begin{pmatrix}\cos(2\pi/n) & \sin(2\pi/n)\\
-\sin(2\pi/n) & \cos(2\pi/n)\end{pmatrix},
$$
и поэтому
$$
x^k = \begin{pmatrix}\cos(2\pi k/n) & \sin(2\pi k/n)\\
-\sin(2\pi k/n) & \cos(2\pi k/n)\end{pmatrix}.
$$
Удобно считать, что вершины нашего многоугольника~--- это в точности
корни степени $n$ из единицы
(см. замечание~\ref{rem:roots_of_unity_geometry}):
$1,\eps,\eps^2,\dots,\eps^{n-1}$.
Тогда одна из осевых симметрий, лежащих в $D_n$~--- это просто
комплексное сопряжение; обозначим эту симметрию через $y$:
$$
y = \begin{pmatrix} 1 & 0\\
0 & -1\end{pmatrix}.
$$
Группа $D_n$ также должна содержать элементы вида $yx^k$ для
$k=1,\dots,n-1$:
$$
yx^k = \begin{pmatrix}\cos(2\pi k/n) & \sin(2\pi k/n)\\
\sin(2\pi k/n) & -\cos(2\pi k/n)\end{pmatrix}.
$$

Теперь можно забыть про школьную геометрию и определить группу $D_n$
как множество, состоящее из матриц $x^k$ и $yx^k$, где
$k=0,\dots,n-1$.

\begin{theorem}
Множество $D_n = \{x^k\mid 0\leq k\leq n-1\}\cup\{yx^k\mid 0\leq k\leq
n-1\}$ (матрицы $x$, $y$ указаны выше) является группой относительно
обычного умножения матриц (и, таким образом, подгруппой в $\GL(2,\mb
R)$). Группа $D_n$ порождена двумя элементами $x$ и $y$;
$\ord_{D_n}(x)=n$, $\ord_{D_n}(y)=2$. Подгруппа $\la x\ra\leq D_n$
циклическая порядка $n$; она нормальна в $D_n$.
\end{theorem}
\begin{proof}
Прямое вычисление показывает, что $x^n=1$ и $y^2=1$; более того,
порядок $x$ равен $n$. Показатель степени $x$ теперь можно
воспринимать по модулю $n$: $x^m = x^{m\mmod n}\in D_n$.
Кроме того, $yxy = x^{-1}$, откуда $xy =
yx^{-1}$ и, итерируя, получаем $x^ky = yx^{-k}$.
Поэтому $x^k\cdot x^l = x^{k+l}$, 
$yx^k\cdot x^l = yx^{k+l}$,
$x^k\cdot yx^l = yx^{-k}x^l = yx^{l-k}$,
$yx^k\cdot yx^l = yyx^{-k}x^l = x^{l-k}$.
Наконец, отсюда следует, что $(x^k)^{-1} = x^{-k}$ и
$(yx^k)^{-1} = yx^k$.
Мы получили, что умножение и взятие обратного не выводит нас за
пределы множества $D_n$; поэтому $D_n\leq\GL(2,\mb R)$. В частности,
$D_n$ является группой. По определению каждый элемент $D_n$ записан в
виде произведения некоторого количества элементов $x$ и $y$, поэтому
$D_n = \la x,y\ra$. Из того, что
$\ord_{D_n}(x) = n$, следует, что $\la x\ra$~--- циклическая порядка
$n$. Наконец, $yx^l\cdot x^k\cdot (yx^l)^{-1} =
yx^l\cdot x^k\cdot yx^l = yx^l\cdot yx^{l-k}=x^{l-k-l} = x^{-k}\in\la
x\ra$, поэтому $\la x\ra\trleq D_n$ (впрочем, нормальность следует и
из примера~\ref{examples:normal_subgroups}
(\ref{item:normal_subgroup_of_index_2}): $\la x\ra$ имеет индекс
$2$ в $D_n$).
\end{proof}

\begin{remark}
Обозначим $\la y\ra = G$, $\la x\ra = H$. Тогда $D_n = GH$: любой
элемент $D_n$ можно записать (и даже единственным образом) в виде
$gh$, где $g\in G$, $h\in H$. Кроме того, $G\cap H = \{e\}$. Более
того, группа $D_n/H$ состоит из двух элементов, потому она циклическая
(теорема~\ref{thm:groups_of_prime_order}) и изоморфна $G$. Однако, $D_n$ не является прямым
произведением $G$ и $H$ (при $n>2$): не хватает
условия~\ref{item:they_commute} из
теоремы~\ref{thm:direct_product}.
Еще один аргумент: подгруппа $G=\la y\ra$ не нормальна
в $D_n$ ($xyx^{-1} = yx^{-2}\notin \la y\ra$) а сомножители должны
быть нормальны в прямом произведении
(предложение~\ref{prop:direct_product_properties},
пункт~\ref{item:direct_product_2}).
\end{remark}
