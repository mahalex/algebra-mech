\section{Эвклидовы и унитарные пространства}

\subsection{Эвклидовы пространства}

\literature{[F], гл. XIII, \S~1, п. 1; [K2], гл. 3, \S~1, п. 1; [KM,
  ч. 2, \S~2, пп. 1--3; \S~5, п. 1.}

\begin{definition}\label{def:bilinear_form}
Пусть $V$~--- векторное пространство над полем $k$. Отображение
$B\colon V\times V\to k$ называется \dfn{билинейной
  формой}\index{билинейная форма}, если оно линейно по каждому
аргументу. Иными словами,
\begin{align*}
&B(u_1+u_2,v) = B(u_1,v) + B(u_2,v),\\
&B(u\alpha,v) = B(u,v)\alpha,\\
&B(u,v_1+v_2) = B(u,v_1) + B(u,v_2),\\
&B(u,v\alpha) = B(u,v)\alpha
\end{align*}
для всех $u,v,u_1,u_2,v_1,v_2\in V$ и $\alpha\in k$.
Если $B(u,v)=0$, то говорят, что вектор $u$
\dfn{ортогонален}\index{ортогональные векторы} вектору $v$
относительно формы $B$. Обозначение: $u\perp v$.
\end{definition}

\begin{definition}
Форма $B$ называется \dfn{симметрической}, если $B(u,v) = B(v,u)$ для
всех $u,v\in V$. Форма $B$ называется \dfn{кососимметрической}, если
$B(u,v) = - B(v,u)$ для всех $u,v\in V$. Форма $B$ называется
\dfn{симплектической}, если $B(u,u) = 0$
для всех $u\in V$.
\end{definition}

\begin{remark}
Симплектическая форма является кососимметрической. Действительно, для
любых $u,v\in V$ тогда выполнено $0 = B(u+v,u+v) = B(u,u) + B(u,v) +
B(v,u) + B(v,v) = B(u,v) + B(v,u)$.
Обратное, вообще говоря, неверно. В самом деле, из кососимметричности
формы сразу следует, что $B(u,u) = - B(u,u)$, откуда $2B(u,u) = 0$ для
всех $u\in V$. Если характеристика поля $k$ не равна $2$, то $2\in
k^*$ и каждая кососимметрическая форма является симплектической. Если
же $k$~--- поле характеристики $2$, то эти два класса форм не
совпадают.
\end{remark}

\begin{example}
В эвклидовом пространстве $V=\mb R^n$ над полем $\mb R$ определены
длины векторов и углы между векторами. Поэтому естественно определить
{\it эвклидово скалярное произведение} формулой $(u,v) = |u|\cdot
|v|\cdot\cos(\ph)$, где $|u|$, $|v|$~--- длины векторов $u$, $v$
соответственно, а $\ph$~--- угол между векторами $u$ и $v$.
Это скалярное произведение симметрично и для любого вектора $v\in V$
выполнено $(v,v)\geq 0$. Более того, равенство $(v,v)=0$ выполнено
только для $v=0$.
\end{example}

Нас интересует алгебра, поэтому мы будем пользоваться чисто
алгебраическими определениями билинейных форм, не ссылающимися на
понятия <<длины>> и <<угла>>; наоборот, чуть позже мы
{\it определим} слова <<длина>> и <<угол>> в терминах билинейных форм.

\begin{example}\label{example:standard_bilinear_form}
Пусть $k$~--- произвольное поле, $V=k^n$~--- пространство столбцов
высоты $n$ над $k$. Определим форму $B\colon V\times V\to k$ формулой
$B(u,v) = u_1v_1 + \dots + u_nv_n$. Иными словами, $B(u,v) = u^Tv$.
Нетрудно видеть, что эта форма билинейна
\begin{align*}
&B(u_1+u_2,v) = (u_1+u_2)^Tv = u_1^Tv + u_2^Tv = B(u_1,v) + B(u_2,v)\\
&B(u\lambda,v)=(u\lambda)^Tv=\lambda(u^Tv)=\lambda B(u,v)\\
&B(u,v_1+v_2) = u^T(v_1+v_2) = u^Tv_1 + u^Tv_2 = B(u,v_1) + B(u,v_2)\\
&B(u,v\lambda)=u^T(v\lambda)=\lambda(u^Tv)=\lambda B(u,v)
\end{align*}
и симметрична
$$
B(u,v) = B(u,v)^T = (u^Tv)^T = v^Tu = B(v,u).
$$
\end{example}

Возьмем теперь в предыдущем примере в качестве $k$ поле вещественных
чисел $\mb R$. Заметим, что скалярное произведение вектора на себя
является неотрицательным числом: $B(u,u) = u_1^2 + \dots + u_n^2\geq
0$; более того, $B(u,u) = 0$ только для $u=0$.

\begin{definition}
Пусть $V$~--- векторное пространство над $\mb R$. Билинейная форма
$B\colon V\times V\to\mb R$ называется \dfn{неотрицательно
  определенной}\index{форма!неотрицательно определенная}, если
$B(u,u)\geq 0$ для всех $u\in V$. Форма $B$
называется \dfn{положительно
  определенной}\index{форма!положительно определенная}, если она
неотрицательно определена и из $B(u,u)=0$ следует, что $u=0$.
\end{definition}

\begin{definition}
Векторное пространство $V$ над полем $\mb R$ вместе с положительно
определенной симметрической билинейной формой $B\colon V\times V\to\mb
R$ называется \dfn{эвклидовым
  пространством}\index{пространство!эвклидово}, а форма $B$ называется
\dfn{эвклидовым скалярным произведением} на $V$.
\end{definition}

\begin{remark}\label{rem:euclidean_subspace}
Любое подпространство $W\leq V$ эвклидова пространства $(V,B)$ само
является эвклидовым пространством относительно скалярного произведения
$B|_{W\times W}\colon W\times W\to\mb R$, которое мы часто будем
обозначать той же буквой $B$. Действительно, нетрудно проверить, что
$B|_{W\times W}$~--- симметрическая билинейная форма, и положительная
определенность формы $B|_{W\times W}$ сразу следует из положительной
определенности формы $B$.
\end{remark}

\subsection{Унитарные пространства}

\literature{[F], гл. XIII, \S~1, пп. 1, 3, [K2], гл. 3, \S~2, п. 2;
  [KM], ч. 2, \S~2, пп. 1--3; \S~6, п. 1.}

В связи с возникновением квантовой механики в первой половине XX века
большое практическое значение стало придаваться векторным
пространствам над полем комплексных чисел $\mb C$.
Что будет аналогом положительно определенных билинейных форм в этом
случае? Заметим, что прямой перенос определения на комплексный случай
не работает: если $V$~--- векторное пространство над полем $\mb C$ и
$B\colon V\times V\to\mb C$~--- билинейная форма, то
$B(iv,iv) = -B(v,v)$ для всех $v\in V$.

\begin{definition}
Отображение $B\colon V\times V\to\mb C$ называется
\dfn{полуторалинейной формой}\index{форма!полуторалинейная}, если оно
{\it линейно} по второму аргументу и
{\it полулинейно} по первому аргументу:
\begin{align*}
&B(u,v_1+v_2) = B(u,v_1) + B(u,v_2)\\
&B(u,v\lambda) = B(u,v)\lambda\\
&B(u_1+u_2,v) = B(u_1,v) + B(u_2,v)\\
&B(u\lambda,v) = \ol\lambda B(u,v)
\end{align*}
для всех $u,v,u_1,u_2,v_1,v_2\in V$ и всех $\lambda\in\mb C$.
\end{definition}

Аналог условия симметричности формы также должен отличаться от
билинейного случая, поскольку теперь $B(u,v\lambda)=\lambda B(u,v)$,
но $B(v\lambda,u) = \ol\lambda B(v,u)$.

\begin{definition}
Полуторалинейная форма $B\colon V\times V\to\mb C$ называется
\dfn{эрмитовой}\index{форма!эрмитова}, если для всех $u,v\in V$
выполнено $B(u,v) = \overline{B(v,u)}$.
\end{definition}

\begin{remark}\label{rem:hermitian_square_is_real}
Заметим, что если $B$~--- эрмитова форма на $V$, то $B(u,u) =
\ol{B(u,u)}$ для всех $u\in V$, поэтому $B(u,u)$~--- вещественное число.
\end{remark}

\begin{example}\label{example:standard_sesquilinear_form}
Пусть  $V=\mb C^n$~--- пространство столбцов
высоты $n$ над $k$. Определим форму $B\colon V\times V\to\mb C$
формулой $B(u,v) = \ol{u_1}v_1 + \dots + \ol{u_n}v_n$. Иными словами,
$B(u,v) = \ol{u}^Tv$. 
Нетрудно видеть, что эта форма полуторалинейная
\begin{align*}
&B(u,v_1+v_2) = \ol{u}^T(v_1+v_2) = \ol{u}^Tv_1 + \ol{u}^Tv_2 = B(u,v_1) +
B(u,v_2)\\
&B(u,v\lambda)=\ol{u}^T(v\lambda)=\lambda(\ol{u}^Tv)=\lambda B(u,v)\\
&B(u_1+u_2,v) = \ol{(u_1+u_2)}^Tv = \ol{u_1}^Tv + \ol{u_2}^Tv = B(u_1,v)
+ B(u_2,v)\\
&B(u\lambda,v)=\ol{(u\lambda)}^Tv=\ol\lambda(\ol{u}^Tv)=\ol\lambda B(u,v)\\
\end{align*}
и эрмитова
$$
\ol{B(u,v)} = \ol{B(u,v)}^T = \ol{(\ol{u}^Tv)}^T = \ol{v^T\ol{u}} =
\ol{v}^Tu = B(v,u).
$$
Заметим, что $B(u,u) = \ol{u_1}u_1 + \dots + \ol{u_n}u_n
= |u_1|^2 + \dots + |u_n|^2 \geq 0$; более того, $B(u,u) = 0$ только
для $u=0$.
\end{example}

\begin{definition}
Пусть $V$~--- векторное пространство над $\mb C$. Эрмитова
форма $B\colon V\times V\to\mb C$ называется \dfn{неотрицательно
  определенной}\index{форма!неотрицательно определенная}, если
$B(u,u)\geq 0$ для всех $u\in V$. Форма $B$
называется \dfn{положительно
  определенной}\index{форма!положительно определенная}, если она
неотрицательно определена и из $B(u,u)=0$ следует, что $u=0$.
\end{definition}

\begin{definition}
Векторное пространство $V$ над полем $\mb C$ вместе с положительно
определенной эрмитовой формой $B\colon V\times V\to\mb
C$ называется \dfn{унитарным
  пространством}\index{пространство!унитарное}, а форма $B$ называется
\dfn{эрмитовым скалярным произведением} на $V$.
\end{definition}

\begin{remark}
Как и в эвклидовом случае
(см. замечание~\ref{rem:euclidean_subspace}), любое подпространство
$W\leq V$ унитарного
пространства $(V,B)$ само 
является унитарным пространством относительно скалярного произведения
$B|_{W\times W}\colon W\times W\to\mb C$, которое мы часто будем
обозначать той же буквой $B$.
\end{remark}

В дальнейшем мы будем параллельно развивать теорию эвклидовых и
унитарных пространств; мы будем обозначать через $k$ поле $\mb R$ или
$\mb C$. Заметим, что и для эвклидовых, и для унитарных пространств
выполнены тождества $B(u,v\lambda) = B(u,v)\lambda$ и $B(u\lambda,v) =
\ol\lambda B(u,v)$; отличие лишь в том, что для эвклидовых пространств
константа $\lambda$ является вещественной, поэтому $\ol\lambda =
\lambda$. Кроме того, условия симметричности и эрмитовости также можно
записать в единообразном виде: $B(u,v) = \ol{B(v,u)}$.


\subsection{Норма}

\literature{[F], гл. XII, \S~1, пп. 1--3, [K2], гл. 3, \S~1, п. 2;
  \S~2, п. 2; [KM], ч. 2, \S~2, п. 4; \S~5, пп. 2--5; \S~6, пп. 4--7.}

\begin{definition}
Пусть $(V,B)$~--- эвклидово или унитарное пространство, $v\in
V$. Будем называть число
$||v|| = \sqrt{B(v,v)}$ \dfn{длиной}\index{длина вектора} $v$.
\end{definition}

\begin{lemma}\label{lem:triangle_inequality}
Пусть $(V,B)$~--- эвклидово или унитарное пространство, $u,v,\in V$. Тогда
\begin{enumerate}
\item ({\it Однородность нормы}). $||v\lambda|| = |\lambda|\cdot
  ||v||$ для любого $\lambda\in k$.
\item ({\it Теорема Пифагора}). Если $B(u,v)=0$, то $||u+v||^2 = ||u||^2
  + ||v||^2$.
\item ({\it Неравенство Коши--Буняковского--Шварца}).
$|B(u,v)|\leq ||u||\cdot ||v||$, причем равенство достигается тогда и
только тогда, когда векторы $u$ и $v$ пропорциональны.
\item ({\it Неравенство треугольника}). $||u||+||v||\geq ||u+v||$;
\end{enumerate}
\end{lemma}
\begin{proof}
Заметим, что для $v=0$ все утверждения леммы очевидны. Поэтому далее
мы будем считать, что $v\neq 0$.

Однородность нормы следует из полуторалинейности:
$$
||v\lambda||^2 = B(v\lambda, v\lambda ) =
\lambda\ol{\lambda}B(v,v) = |\lambda|^2\cdot ||v||^2.
$$

Заметим, что $||u+v||^2 = B(u+v,u+v) = B(u,u) + B(u,v) +
\ol{B(u,v)} + B(v,v)$, и при $B(u,v)=0$ получаем в точности теорему
Пифагора.

Для доказательства неравенства Коши--Буняковского--Шварца положим
$$
w = u - v\frac{B(u,v)}{B(v,v)}
$$
и заметим, что $$B(w,v) = B(u-v\frac{B(u,v)}{B(v,v)},v)
 = B(u,v) - \frac{B(u,v)}{B(v,v)}B(v,v) = 0.$$
Это означает, что векторы $v$ и $w$ ортогональны. Поэтому и вектор
$v\frac{B(u,v)}{B(v,v)}$ ортогонален вектору $w$. Применим к этой паре
векторов теорему Пифагора:
$$
||u||^2 = ||w||^2 + ||v\frac{B(u,v)}{B(v,v)}||^2 = ||w||^2 +
\frac{|B(u,v)|^2}{||v||^2} \geq \frac{|B(u,v)|^2}{||v||^2},
$$
откуда $|B(u,v)|\leq ||u||\cdot ||v||$.
Если достигается равенство, то $||w||=0$, откуда $w=0$ и $u$
пропорционально $v$; обратно, если $u$ пропорционально $v$, то
в неравенстве Коши--Буняковского--Шварца имеет место равенство.

Посмотрим на выражение для $B(u+v,u+v)$:
\begin{align*}
||u+v||^2 &= B(u+v,u+v)\\
&= B(u,u) + B(u,v) + \ol{B(u,v)}+ B(v,v)\\
&= ||u||^2 + 2\Ree(B(u,v)) + ||v||^2 \leq ||u||^2 + 2|B(u,v)| + ||v||^2\\
&\leq ||u||^2 +2||u||\cdot ||v|| + ||v||^2\\
&= (||u||+||v||)^2.
\end{align*}
Извлекая корень из обеих частей, получаем неравенство треугольника.
\end{proof}

\begin{definition}
Пусть $(V,B)$~--- эвклидово пространство.
Лемма~\ref{lem:triangle_inequality} показывает, что для ненулевых
векторов $u,v\in V$ выражение $\frac{B(u,v)}{||u||\cdot ||v||}$ лежит
на отрезке $[-1,1]$ и потому является косинусом некоторого однозначно
определенного угла $\ph\in [0,\pi]$. Этот угол называется \dfn{углом
  между векторами}\index{угол между векторами} $u$ и $v$. Обозначение:
$\ph = \angle(u,v)$. Обратите внимание, что это определение не
работает для унитарного пространства: $B(u,v)$ может оказаться
комплексным. Однако, имеет смысл рассматривать выражение
$\frac{|B(u,v)|}{||u||\cdot ||v||}$; оно лежит на отрезке $[0,1]$ и
потому является косинусом некоторого однозначно определенного угла
$\ph\in[0,\frac{\pi}{2}]$.
\end{definition}

\begin{remark}
Заметим, что угол $\angle(u,v)$ равен $\pi/2$ тогда и только тогда,
когда $B(u,v)=0$, то есть, когда векторы $u$ и $v$ ортогональны в смысле
определения~\ref{def:bilinear_form}.
\end{remark}


\subsection{Матрица Грама}

\literature{[F], гл. XIII, \S~1, п. 4; [KM], ч. 2, \S~2, пп. 2--3;
  [KM], ч. 2, \S~3, п. 8.}

Пусть $(V,B)$~--- конечномерное пространство над полем $k$ с формой,
билинейной в
случае $k=\mb R$ и полуторалинейной в случае $k=\mb C$. Пусть
$\mc E = (e_1,\dots,e_n)$~--- базис $V$.
Запишем векторы $u,v\in V$ в этом базисе:
$u = e_1u_1 + \dots + e_nu_n$,
$v = e_1v_1 + \dots + e_nv_n$.
Подставим эти выражения в $B(u,v)$:
$$
B(u,v) = B(e_1u_1+\dots+e_nu_n, e_1v_1+\dots+e_nv_n)
= \sum_{i,j=1}^n B(e_iu_i,e_jv_j)
= \sum_{i,j=1}^n \ol{u_i}v_j B(e_i,e_j).
$$
Это означает, что форма $B$ полностью определяется своими значениями
на базисных векторах.
Полученное выражение можно записать в матричной форме:
$$
B(u,v) = \ol{[u]}^T (B(e_i,e_j))_{i,j=1}^n [v],
$$
где через $[u],[v]$ мы обозначаем столбцы координат векторов $u,v$ в
базисе $\mc E$.
Матрица, составленная из скалярных произведений $B(e_i,e_j)$ базисных
векторов, называется
\dfn{матрицей Грама} формы $B$ в базисе $\mc E$.
Обозначим ее через $G$.
Мы получили, что
$B(u,v) = \ol{[u]}^T G [v]$ для всех $u,v\in V$.

Пока мы использовали только билинейность/полуторалинейность формы
$B$. Если форма $B$ симметрична/эрмитова, то
$\ol{B(v,u)} = \ol{B(v,u)}^T = \ol{(\ol{[v]}^T G [u])^T}
= \ol{[u]^T G^T \ol{[v]}} = \ol{[u]}^T \ol{G}^T [v]$. Сравним это с
выражением $B(u,v) = \ol{[u]}^T G [v]$:
$$
\ol{[u]}^T \ol{G}^T [v] = \ol{[u]}^T G [v]\quad\text{ для всех $u,v\in V$}.
$$
Подставляя в качестве $u,v$ базисные векторы $e_1,\dots,e_n$,
получаем, что матрицы $\ol{G}^T$ и $G$ совпадают:
$$
\ol{G}^T = G.
$$
Для случая эвклидова пространства, конечно, это равенство означает,
что $G^T = G$.

\begin{definition}
Матрица $A$ над произвольным полем называется \dfn{симметрической}\index{матрица!симметрическая},
если $A^T = A$. Матрица $A$ над полем комплексных чисел называется
\dfn{эрмитовой}\index{матрица!эрмитова}, если $\ol{A}^T = A$.
\end{definition}

Таким образом, мы показали, что матрица Грама симметрической
билинейной формы является симметрической, а матрица Грама эрмитовой
полуторалинейной формы является эрмитовой.

Обратно, по любой симметрической матрице над $\mb R$ можно построить
симметрическую билинейную форму, а по любой эрмитовой матрице над $\mb
C$~--- эрмитову полуторалинейную форму. Действительно, мы можем
обобщить примеры~\ref{example:standard_bilinear_form}
и~\ref{example:standard_sesquilinear_form}.
Пусть $G\in M(n,k)$~--- симметрическая или эрмитова матрица. На
пространстве столбцов $V=k^n$ высоты $n$ определим форму
$B\colon V\times V\to k$ равенством
$$
B(u,v) = \ol{u}^TGv.
$$
Нетрудно проверить, что эта форма билинейна в случае $k=\mb R$ и
полуторалинейна в случае $k=\mb C$:
\begin{align*}
&B(u,v_1+v_2) = \ol{u}^T G(v_1+v_2) = \ol{u}^TGv_1 + \ol{u}^TGv_2 =
B(u,v_1) + B(u,v_2)\\
&B(u,v\lambda) = \ol{u}^T G(v\lambda) = (\ol{u}^TGv)\lambda = B(u,v)\lambda\\
&B(u_1+u_2,v) = \ol{u_1+u_2}^T Gv = \ol{u_1}^TGv + \ol{u_2}^TGv =
B(u_1,v) + B(u_2,v)\\
&B(u\lambda,v) = \ol{u\lambda}^T Gv = \ol\lambda(\ol{u}^TGv) =
\ol\lambda B(u,v)
\end{align*}
Кроме того, для симметрической матрицы $G$ имеем
$$
B(v,u) = B(v,u)^T = (v^T G u)^T = u^TG^Tv = u^TGv = B(u,v),
$$
а для эрмитовой~---
$$
\ol{B(v,u)} = \ol{B(v,u)}^T = (\ol{\ol{v}^TGu})^T = \ol{u}^T\ol{G}^Tv
= \ol{u}^T G v = B(u,v).
$$
Поэтому форма $B$ является симметрической или эрмитовой
соответственно. По определению исходная матрица $G$ является матрицей
Грама полученной формы $B$ в стандартном базисе пространства столбцов.

Естественно поставить вопрос: как меняется матрица Грама при замене
базиса в пространстве $V$?
Напомним, что если $\mc E=\{e_1,\dots,e_n\}$ и $\mc F=
\{f_1,\dots,f_n\}$~--- два базиса в пространстве $V$, то {\it
  матрица перехода} $(\mc E\rsa\mc F)$ от базиса $\mc E$ к базису
$\mc F$ устроена так:
в столбце с номером $j$ стоят координаты вектора $f_j$ в базисе $\mc E$
(см. определение~\ref{def:change_of_basis_matrix}).

\begin{theorem}[Преобразование матрицы Грама при замене базиса]\label{thm:Gram_matrix_change_of_coordinates}
Пусть $\mc E, \mc F$~--- два базиса конечномерного пространства $V$
над полем $k$, $C = (\mc E\rsa\mc F)$~--- матрица перехода от $\mc E$
к $\mc F$, $B\colon V\times V\to k$~--- билинейная или
полуторалинейная форма на $V$. Пусть $G_{\mc E}$ и $G_{\mc F}$~---
матрицы Грама формы $B$ в базисах 
$\mc E$ и $\mc F$ соответственно.  Тогда
$$
G_{\mc F} = \ol{C}^T G_{\mc E}C.
$$
\end{theorem}

\begin{proof}
Пусть $u,v\in V$. По теореме~\ref{thm:change_of_coordinates}
координаты векторов в базисах $\mc E$, $\mc F$ связаны следующим
образом:
$[v]_{\mc E} = C\cdot [v]_{\mc F}$,
$[u]_{\mc E} = C\cdot [u]_{\mc F}$.
Поэтому
$$
B(u,v) = \ol{[u]_{\mc E}}^T G_{\mc E}[v]_{\mc E} =
\ol{C\cdot[u]_{\mc F}}^T G_{\mc E}C\cdot [v]_{\mc F} =
\ol{[u]_{\mc F}}^T\ol{C}^T G_{\mc E}C\cdot [v]_{\mc F}
$$
С другой стороны,
$$
B(u,v) = \ol{[u]_{\mc F}}^T G_{\mc F}[v]_{\mc F}.
$$
Получаем, что $\ol{[u]_{\mc F}}^T\ol{C}^T G_{\mc E}C\cdot [v]_{\mc F}
= \ol{[u]_{\mc F}}^T G_{\mc F}[v]_{\mc F}$ для всех $u,v\in
V$. Подставляя в качестве $u,v$ всевозможные пары векторов базиса $\mc
F$, получаем необходимое равенство матриц.
\end{proof}

Отметим, что матрица Грама скалярного
произведения обратима.

\begin{proposition}
Пусть $(V,B)$~--- эвклидово или унитарное пространство. Тогда матрица
Грама формы $B$ в любом базисе является обратимой.
\end{proposition}
\begin{proof}
Выберем произвольный базис $\mc E$ пространства $V$ и запишем матрицу
Грама $G=G_{\mc E}\in M(n,k)$ скалярного произведения $B$ в этом
базисе. Если она необратима, то (по теореме
Кронекера--Капелли~\ref{thm_kronecker_kapelli_2}) уравнение
$GX=0$ имеет ненулевое решение: найдется столбец
$X_0\in k^n\setminus\{0\}$, для которого
$GX_0=0$. Такой столбец является столбцом координат некоторого
ненулевого вектора $v_0\in V$. Но тогда
$B(v_0,v_0) = \ol{[v_0]_{\mc E}}^T\cdot G\cdot [v_0]_{\mc E} =
\ol{X_0}^TGX_0 = 0$, что противоречит положительной определенности
формы $B$.
\end{proof}

\subsection{Процесс ортогонализации Грама--Шмидта}

\literature{[F], гл. XIII, \S~1, пп. 5, 6; \S~2, п. 1; [K2], гл. 3,
  \S~1, п. 3; \S~2, п. 3; [KM], ч. 2, \S~3, п. 6; \S~4, пп. 2--4.}

\begin{definition}
Пусть $(V,B)$~--- эвклидово или унитарное пространство.
Базис $(e_1,\dots,e_n)$ пространства $V$ называется
\dfn{ортогональным}\index{базис!ортогональный}, если все его векторы
попарно ортогональны:
$e_i\perp e_j$ при $i\neq j$. Этот базис называется
\dfn{ортонормированным}\index{базис!ортонормированный}, если он
ортогонален и длина каждого вектора равна единице: $||e_i||=1$ для
всех $i$.
\end{definition}

\begin{lemma}\label{lem:orthogonality_implies_independency}
Пусть $(V,B)$~--- эвклидово или унитарное пространство. Если ненулевые
векторы $e_1,\dots,e_n\in V$ попарно ортогональны,
то они линейно независимы. Если, кроме того, $\dim V=n$, то векторы
$e_1,\dots,e_n$ образуют ортогональный базис.
\end{lemma}
\begin{proof}
Предположим, что $e_1\lambda_1 + \dots +
e_n\lambda_n = 0$~--- нетривиальная линейная комбинация этих векторов,
равная нулю. Домножим это равенство скалярно на $e_i$:
$$
B(e_i,e_1\lambda_1 + \dots + e_n\lambda_n) = 0.
$$
Пользуясь линейностью по второму аргументу и попарной ортогональностью
векторов $e_i$, получаем равенство $\lambda_i B(e_i,e_i) = 0$. Так как
$e_i\neq 0$, получаем, что $\lambda_i=0$ для всех $i=1,\dots,n$.

Если $\dim V = n$, мы получаем $n$ линейно независимых векторов в
$n$-мерном векторном пространстве. Из
предложения~\ref{prop:dimension_is_monotonic} следует, что они
образуют базис (действительно, размерность их линейной оболочки
совпадает с размерностью $V$, поэтому эта линейная оболочка равна $V$).
\end{proof}

\begin{remark}
По определению матрица Грама формы $B$ в базисе $\mc E =
(e_1,\dots,e_n)$ составлена из
скалярных произведений $B(e_i,e_j)$. Поэтому базис $\mc E$
ортогонален тогда и только тогда, когда матрица Грама скалярного
произведения в этом базисе диагональна; базис $\mc E$ ортонормирован
тогда и только тогда, когда матрица Грама скалярного произведения в
этом базисе единична.
\end{remark}

Таким образом, если нам дано эвклидово или унитарное пространство,
часто удобно выбрать в нем ортогональный базис: в нем скалярное
произведение задается простыми формулами через координаты векторов
(см. примеры~\ref{example:standard_bilinear_form}
и~\ref{example:standard_sesquilinear_form}: стандартные базисы
пространства столбцов являются ортонормированными относительно
рассматриваемых там форм).

\begin{lemma}[Процесс ортогонализации Грама--Шмидта]\label{lem:Gram_Schmidt}
Пусть $(V,B)$~--- эвклидово или унитарное пространство,
$e_1,\dots,e_{n-1}$~--- семейство попарно ортогональных ненулевых векторов,
$v\notin\la e_1,\dots,e_{n-1}\ra$. Тогда существует вектор $e_n\in V$
такой, что $e_n$ ортогонален всем векторам $e_1,\dots,e_{n-1}$ и,
кроме того, $\la e_1,\dots,e_{n-1},v\ra = \la e_1,\dots,e_{n-1},e_n\ra$.
\end{lemma}
\begin{proof}
Будем искать вектор $e_n$ в виде
$$
e_n = v - e_1\lambda_1 - e_2\lambda_2 - \dots - e_{n-1}\lambda_{n-1}.
$$
Подберем коэффициенты $\lambda_1,\dots,\lambda_{n-1}\in k$ так, чтобы
$e_n$ был ортогонален каждому $e_i$, $i=1,\dots,n-1$. Посмотрим на
скалярное произведение $e_n$ и $e_i$. Поскольку $e_i$ ортогонален
всем векторам из $e_1,\dots,e_{n-1}$, кроме $e_i$, получаем
$$
B(e_i,e_n) = B(e_i,v) - B(e_i,e_i)\lambda_i.
$$
Положим теперь $\lambda_i = \frac{B(e_i,v)}{B(e_i,e_i)}$; заметим, что
$B(e_i,e_i)\neq 0$, поскольку $e_i\neq 0$. Мы добились того, что
$e_n\perp e_i$ для всех $i=1,\dots,n-1$. Кроме того, $v$ выражается
через $e_1,\dots,e_n$, поэтому $v\in\la e_1,\dots,e_n\ra$, и
$e_n$ выражается через $e_1,\dots,e_{n-1},v$, поэтому $e_n\in\la
e_1,\dots,e_{n-1},v\ra$. Это и означает равенство нужных линейных оболочек.
\end{proof}

\begin{corollary}\label{cor:Gram_Schmidt_1}
Пусть $(V,B)$~--- эвклидово или унитарное пространство, и пусть
$\mc F = (f_1,\dots,f_n)$~--- базис $V$. Тогда существует
ортогональный базис $\mc E = (e_1,\dots,e_n)$ пространства $V$ такой,
что $\la e_1,\dots,e_k\ra = \la f_1,\dots,f_k\ra$ для всех $k=1,\dots,n$.
\end{corollary}
\begin{proof}
Индукция по $n$. Для $n=1$ утверждение очевидно: достаточно взять $e_1
= f_1$. Пусть утверждение доказано для всех пространств размерности не
выше $n-1$, и мы взяли пространство $V$ размерности $n$.
Рассмотрим в нашем пространстве $V$ линейную оболочку
векторов $f_1,\dots,f_{n-1}$: $W = \la f_1,\dots,f_{n-1}\ra$. По
предположению индукции найдется ортогональный базис
$e_1,\dots,e_{n-1}$ пространства $W$ такой, что $\la e_1,\dots,e_k\ra
= \la f_1,\dots,f_k\ra$ для всех $k=1,\dots,n-1$.

Применим лемму~\ref{lem:Gram_Schmidt} к набору $e_1,\dots,e_{n-1}$ и
вектору $f_n$. Мы найдем вектор $e_n$ такой, что $e_1,\dots,e_n$~---
ортогональная система векторов, и $\la e_1,\dots,e_n\ra = \la
f_1,\dots,f_n\ra = v$, то есть, $e_1,\dots,e_n$~--- базис
$V$. Очевидно, что условие $\la e_1,\dots,e_k\ra = \la
f_1,\dots,f_k\ra$ теперь выполняется для всех $k=1,\dots,n$.
\end{proof}

\begin{corollary}\label{cor:orthogonal_basis_exists}
В любом [конечномерном] эвклидовом или унитарном пространстве
существует ортогональный (и даже ортонормированный) базис.
\end{corollary}
\begin{proof}
Применим следствие~\ref{cor:Gram_Schmidt_1} к произвольному базису
пространства $V$. Получим ортогональный базис $e_1,\dots,e_n$. Положим
$e'_i = e_i/||e_i||$; легко видеть, что $||e'_i|| = 1$ и векторы
$e'_1,\dots,e'_n$ все еще попарно ортогональны. Мы получили
ортонормированный базис пространства $V$.
\end{proof}

\begin{corollary}\label{cor:orthogonal_basis_extension}
Пусть $V$~--- эвклидово или унитарное пространства, $W\leq V$~---
подпространство в $V$. Любой ортогональный базис подпространства $W$
можно дополнить до ортогонального базиса пространства $V$.
\end{corollary}
\begin{proof}
Как и в доказательстве следствия~\ref{cor:Gram_Schmidt_1},
воспользуемся леммой~\ref{lem:Gram_Schmidt} для индуктивного
построения нужного базиса.
\end{proof}

\subsection{Ортогональные и унитарные матрицы}

\literature{[F], гл. XIII, \S~1, п 7; [K2], гл. 3, \S~1, п. 5; \S~2,
  п. 4.}

В этом разделе мы выясним, что матрица перехода между ортогональными
базисами является ортогональной в эвклидовом случае и унитарной в
унитарном случае.

\begin{definition}
Матрица $C\in M(n,\mb R)$ называется
\dfn{ортогональной}\index{матрица!ортогональная}, если $C\cdot C^T =
C^T\cdot C = E$. Матрица $C\in M(n,\mb C)$ называется
\dfn{унитарной}\index{матрица!унитарная}, если $C\cdot \ol{C}^T =
\ol{C}^T\cdot C = E$.
\end{definition}

\begin{remark}
Конечно, условия ортогональности и унитарности матрицы записываются
единообразно ($C\cdot\ol{C}^T=\ol{C}^T\cdot C=E$), если помнить, что
$\ol{C}=C$ для $C\in M(n,\mb R)$.
\end{remark}

\begin{lemma}\label{lem:orthogonal_equivalencies}
Для матрицы $C\in M(n,\mb R)$ следующие условия равносильны:
\begin{enumerate}
\item $C$ ортогональна
\item $C^T$ ортогональна
\item столбцы $C$ образуют ортонормированный базис в
  эвклидовом пространстве $\mb R^n$ со стандартным эвклидовым
  скалярным произведением
  (пример~\ref{example:standard_bilinear_form}).
\item строки $C$ образуют ортонормированный базис в эвклидовом
  пространстве ${}^n\mb R$ со стандартным эвклидовым скалярным
  произведением.
\end{enumerate}
\end{lemma}

\begin{lemma}\label{lem:unitary_equivalencies}
Для матрицы $C\in M(n,\mb C)$ следующие условия равносильны:
\begin{enumerate}
\item $C$ унитарна
\item $\ol{C}^T$ унитарна
\item столбцы $C$ образуют ортонормированный базис в унитарном
  пространстве $\mb C^n$ со стандартным эрмитовым скалярным
  произведением (пример~\ref{example:standard_sesquilinear_form}).
\item строки $C$ образуют ортонормированный базис в унитарном
  пространстве ${}^n\mb C$ со стандартным эрмитовым скалярным
  произведением.
\end{enumerate}
\end{lemma}

\begin{proof}
Мы докажем только вариант для унитарной матрицы.
\begin{itemize}
\item[$(1)\Leftrightarrow (2)$] Очевидно из определения.
\item[$(1)\Rightarrow (3)$] Посмотрим на равенство $\ol{C}^T\cdot
  C=E$. Оно означает, что при умножении $i$-ой строки матрицы
  $\ol{C}^T$ на $j$-й столбец матрицы $C$ мы получим
  $\delta_{ij} = \begin{cases}1,&i=j,\\0,&i\neq j.\end{cases}$. То
  есть, при стандартном эрмитовом скалярном произведении $i$-го
  столбца матрицы $C$ на ее $j$-й столбец получается $\delta_{ij}$. Это
  означает, что столбцы матрицы $C$ попарно ортогональны и, кроме того,
  длина каждого столбца равна $1$. В частности, все столбцы
  ненулевые. По лемме~\ref{lem:orthogonality_implies_independency} эти
  столбцы образуют ортонормированный базис в $\mb C^n$.
\item[$(3)\Rightarrow (1)$] Мы знаем, что стандартное эрмитово
  скалярное произведение $i$-го столбца матрицы $C$ на ее $j$-й
  столбец равно $\delta_{ij}$. Но в точности это произведение стоит в
  позиции $(i,j)$ матрицы $\ol{C}^T\cdot C$; поэтому $\ol{C}^T\cdot C
  = E$. Заметим, что $1 = \det(E) = \det(\ol{C}^T\cdot C) =
  \ol\det(C)\cdot\det(C)$, поэтому $\det(C)$ отличен от нуля и, стало
  быть, матрица $C$ обратима. Из равенства $\ol{C}^T\cdot C = E$
  теперь следует, что $C^{-1} = \ol{C}^T$, и поэтому $C\cdot\ol{C}^T =
  E$.
\item[$(2)\Leftrightarrow (4)$] Применим только что доказанную
  равносильность $(1)\Leftrightarrow (3)$ к матрице $C^T$; осталось
  только заметить, что сопряжение не меняет выполнение свойства $(3)$:
  если $e_1,\dots,e_n$~--- ортонормированный базис унитарного
  пространства $\mb C^n$, то и $\ol{e_1},\dots,\ol{e_n}$~---
  ортонормированный базис того же пространства.
\end{itemize}
\end{proof}

\begin{theorem}
Пусть $(V,B)$~--- эвклидово или унитарное пространство.
Пусть $\mc E$, $\mc F$~--- ортонормированные базисы $V$, и
$C=(\mc E\rsa\mc F)$~--- матрица перехода между ними. Тогда матрица
$C$ ортогональна в случае эвклидова пространства и унитарна в случае
унитарного пространства.
\end{theorem}
\begin{proof}
По теореме~\ref{thm:Gram_matrix_change_of_coordinates} выполнено
$G_{\mc F} = \ol{C}^T\cdot G_{\mc E}\cdot C$, где
$G_{\mc E}$, $G_{\mc F}$~--- матрицы Грама формы $B$ в базисах $\mc E$,
$\mc F$ соответственно. Но базисы $\mc E$, $\mc F$ ортонормированы,
поэтому $G_{\mc E} = G_{\mc F} = E$. Значит, $E = \ol{C}^T\cdot C$, и
матрица $C$ ортогональна в эвклидовом случае и унитарна в унитарном
случае.
\end{proof}

\subsection{Ортонормированные базисы}

Введенное выше понятие ортонормированного базиса чрезвычайно полезно:
в этом разделе мы увидим, что использование таких базисов упрощает вычисления.

\begin{lemma}\label{lem:orthonormal-basis-coordinates}
Пусть $(V,B)$~--- эвклидово или унитарное пространство,
$e_1,\dots,e_n$~--- ортонормированный базис $V$,
$v\in V$~--- произвольный вектор, и $v = e_1\alpha_1 + \dots + e_n\alpha_n$~---
его разложение по этому базису.
Тогда $\alpha_i = B(e_i,v)$ и
$||v||^2 = |\alpha_1|^2 + \dots + |\alpha_n|^2$.
\end{lemma}
\begin{proof}
Домножим равенство $v = e_1\alpha_1 + \dots + e_n\alpha_n$
скалярно на $e_i$:
$$
B(e_i,v) = B(e_i, e_1\alpha_1 + \dots + e_n\alpha_n).
$$
Воспользовавшись линейностью $B$ по второму аргументу и ортонормированностью
базиса $e_1,\dots,e_n$, получаем, что $B(e_i,v) = B(e_i,e_i\alpha_i) = \alpha_i$.
Заметим, что векторы $e_1\alpha_1,\dots,e_n\alpha_n$ попарно ортогональны и
$||e_i\alpha_i|| = |\alpha_i|$. Доказательство завершается индукцией по $n$
с применением теоремы Пифагора.
\end{proof}

Пусть $(V,B)$~--- конечномерное эвклидово или унитарное пространство,
$u\in V$~--- некоторый фиксированный вектор. Рассмотрим отображение
$B(u,{-})\colon V\to k$, $v\mapsto B(u,v)$. Линейность формы $B$ по второму
аргументу означает, что полученное отображение линейно, то есть,
лежит в $\Hom_k(V,k)$. Оказывается, верно и обратное: любое линейное отображение
из $V$ в основное поле $k$ имеет вид $B(u,{-})$ для некоторого вектора $u\in V$.

Заметим, что если фиксированный вектор $u$ поставить на второе место, то
мы получим {\em полулинейное} отображение $B({-},u)\colon V\to k$
(оно обладает свойством аддитивности, а скаляр выносится с сопряжением). Аналогично,
любое полулинейное отображение из $V$ в $k$ имеет вид $B({-},u)$
для некоторого вектора $u\in V$.

\begin{theorem}[Теорема Риса]\label{thm:Riesz_theorem}
Пусть $(V,B)$~--- конечномерное эвклидово или унитарное пространство.
Если $\ph\colon V\to k$~--- линейное отображение, то существует
единственный вектор $u\in V$ такой, что $\ph(v) = B(u,v)$ для всех $v\in V$.
Если $\ph\colon V\to k$~--- полулинейное отображение, то существует
единственный вектор $u\in V$ такой, что $\ph(v) = B(v,u)$ для всех $v\in V$.
\end{theorem}
\begin{proof}
Пусть $\ph\colon V\to k$~--- линейное отображение.
Выберем некоторый ортонормированный базис $e_1,\dots,e_n$ пространства $V$.
Пусть $v\in V$~--- произвольный вектор.
Тогда по лемме~\ref{lem:orthonormal-basis-coordinates}
$$
v = e_1 B(e_1,v) + e_2 B(e_2,v) + \dots + e_n B(e_n,v).
$$
Применяя к этому равенству отображение $\ph$ и пользуясь его линейностью, получаем
\begin{align*}
\ph(v) &= \ph(e_1 B(e_1,v) + e_2 B(e_2, v) + \dots + e_n B(e_n,v)) \\
&= \ph(e_1)B(e_1,v) + \ph(e_2)B(e_2,v) + \dots + \ph(e_n)B(e_n) \\
&= B(e_1\overline{\ph(e_1)} + e_2\overline{\ph(e_2)} + \dots + e_n\overline{\ph(e_n)},v).
\end{align*}
Заметим, что первый аргумент полученного выражения не зависит от $v$.
Положив $u = e_1\overline{\ph(e_1)} + e_2\overline{\ph(e_2)} + \dots
+ e_n\overline{\ph(e_n)}$, получаем,
что $\ph(v) = B(u,v)$ для произвольного $v\in V$. Осталось показать, что такой
вектор $u$ единственный. Предположим, что нашелся еще один вектор $u'\in V$
такой, что $\ph(v) = B(u',v)$ для всех $v\in V$.
Но тогда $B(u,v) = \ph(v) = B(u',v)$, откуда $B(u-u',v) = 0$ для всех $v\in V$.
В частности, это так для $v = u-u'$, и получаем $B(u-u',u-u') = 0$.
Но форма $B$ положительно определена, и потому $u-u'=0$, то есть, $u=u'$.

Пусть теперь отображение $\ph\colon V\to k$ полулинейно. Тогда
отображение $\overline\ph\colon V\to k$, $v\mapsto \overline{\ph(v)}$,
линейно, и к нему можно применить доказанное выше: существует единственный вектор
$u\in V$ такой, что $\overline\ph(v) = B(u,v)$ для всех $u\in V$.
Но равенство $\overline\ph(v) = B(u,v)$ равносильно равенству
$\ph(v) = B(v,u)$.
\end{proof}

\begin{remark}
Заметим, что полученное выражение
$u = e_1\overline{\ph(e_1)} + \dots + e_n\overline{\ph(e_n)}$
для вектора $u$ с виду зависит от выбора базиса $e_1,\dots,e_n$.
С другой стороны, мы показали, что вектор $u$ с указанными свойствами
единственный. Получается, что это выражение на самом деле одинаково
во всех базисах пространства $V$.
\end{remark}

\subsection{Ортогональное дополнение}

\literature{[F], гл. XIII, \S~2, п. 2; [K2], гл. 3, \S~1, п. 3; \S~2,
  п. 3; [KM], ч. 2, \S~3, пп. 1--2.}

\begin{definition}
Пусть $(V,B)$~--- эвклидово или унитарное пространство, $U\subseteq V$~---
произвольное подмножество.
\dfn{Ортогональным дополнением}\index{ортогональное дополнение} к подмножеству
$U$ в $V$ называется
$U^\perp = \{v\in V\mid \forall u\in U\;\; B(u,v) = 0\}$.
\end{definition}

\begin{proposition}\label{prop:orthogonal-complement-properties}
Пусть $(V,B)$~--- эвклидово или унитарное пространство,
$U\subseteq V$~--- подмножество в $V$. Тогда
\begin{enumerate}
\item $U^\perp$ является подпространством в $V$;
\item $\{0\}^\perp = V$, $V^\perp = \{0\}$;
\item $U\cap U^\perp \subseteq\{0\}$;
\item если $U\subseteq W$~--- два подмножества в $V$, то $W^\perp\subseteq U^\perp$.
\end{enumerate}
\end{proposition}
\begin{proof}
\begin{enumerate}
\item Если $v_1,v_2$ лежат в $U^\perp$, то для любого $u\in U$ выполнено
  $B(u,v_1) = B(u,v_2) = 0$. Поэтому для любых $\lambda_1,\lambda_2\in
  k$ выполнено $B(u,v_1\lambda_1+v_2\lambda_2) = B(u,v_1)\lambda_1 +
  B(u,v_2)\lambda_2 = 0$, и $v_1\lambda_1+v_2\lambda_2\in
  U^\perp$. Это доказывает, что $U^\perp\leq V$.
\item Любой вектор $V$ ортогонален $0$, поэтому $\{0\}^\perp = V$. Если
  вектор $v\in V$ ортогонален всем векторам из $V$, то, в частности,
  он ортогонален самому себе, то есть, $B(v,v)=0$. В силу
  положительной определенности формы $B$ из этого следует, что
  $v=0$. Это доказывает, что $V^\perp = \{0\}$.
\item Пусть $v\in U\cap U^\perp$. Условие $v\in U^\perp$ означает,
  что $B(u,v) = 0$ для всех $u\in U$, в частности, для $u=v$.
  Поэтому $B(v,v)=0$. В силу положительной определенности формы $B$
  получаем, что $v=0$.
\item Пусть $v\in W^\perp$. Тогда $B(u,v) = 0$ для всех $u\in W$. В частности,
  это так для всех $u\in U$. Поэтому $v\in U^\perp$.
\end{enumerate}
\end{proof}

\begin{proposition}\label{prop:orthogonal-complement-properties-findim}
Пусть $(V,B)$~--- эвклидово или унитарное пространство,
$U\leq V$~--- конечномерное подпространство в $V$. Тогда
\begin{enumerate}
\item\label{num:orth-comp-prop-findim-1} $V = U\oplus U^\perp$;
\item если, кроме того, $V$ конечномерно, то $\dim (U^\perp) = \dim (V) - \dim (U)$;
\item $(U^\perp)^\perp = U$.
\end{enumerate}
\end{proposition}
\begin{proof}
\begin{enumerate}
\item Пусть $e_1,\dots,e_m$~--- некоторый ортонормированный базис
  подпространства $U$ (такой существует по
  следствию~\ref{cor:orthogonal_basis_exists}).
  Возьмем произвольный вектор $v\in V$, обозначим
  $$
  u = e_1 B(e_1,v) + \dots + e_m B(e_m,v) \in U,
  $$
  и положим $w = v-u$.
  Заметим, что $w\in U^\perp$. Действительно,
  \begin{align*}
  B(e_i,w) &= B(e_i,v-u) \\
  &= B(e_i,v) - B(e_i,u) \\
  &= B(e_i,v) - B(e_i,e_1 B(e_1,v) + \dots + e_m B(e_m,v)) \\
  &= B(e_i,v) - B(e_i,v) \\
  &= 0
  \end{align*}
  (мы воспользовались ортонормированностью базиса $e_1,\dots,e_m$).
  Эта выкладка показывает, что $w$ ортогонален каждому из векторов
  $e_1,\dots,e_m$; поэтому $w$ ортогонален и любой их линейной комбинации,
  то есть, любому вектору подпространства $U$.
  Итак, мы получили представление $v = u + w$, где $u\in U$, $w\in U^\perp$,
  для произвольного вектора $v\in V$. Это означает, что $V = U + U^\perp$.
  В предложении~\ref{prop:orthogonal-complement-properties} мы уже показали,
  что $U\cap U^\perp \subseteq \{0\}$, и в нашем случае $U,U^\perp$ содержат $0$,
  то есть, на самом деле $U\cap U^\perp = \{0\}$.
  По предложению~\ref{prop:direct-sum-criteria-for-2} из этого следует, что
  $V = U\oplus U^\perp$.
\item По следствию \ref{cor:direct-sum-dimension} и по уже доказанному,
  имеем $\dim(V) = \dim(U) + \dim(U^\perp)$.
\item Покажем сначала, что $U\subseteq (U^\perp)^\perp$ (на самом деле, это
  верно даже без условия конечномерности $U$). Пусть $u\in U$; мы хотим проверить,
  что $u\in (U^\perp)^\perp$, то есть, что $u$ ортогонален любому вектору
  из $U^\perp$. Пусть $w$~--- произвольный вектор из $U^\perp$. По определению
  это означает, что он ортогонален любому вектору из $U$, в частности, вектору $u$:
  $B(u,w) = 0$. Но тогда и $B(w,u) = 0$, то есть, $u$ ортогонален $w$, что и
  требовалось.

  Осталось проверить обратное включение: возьмем произвольный вектор
  $v\in (U^\perp)^\perp$ и покажем, что $v\in U$.
  По первому пункту мы можем представить $v$ в виде $v = u + w$,
  где $u\in U$ и $w\in U^\perp$. Тогда $w = v - u$, и отсюда
  $B(w, w) = B(w, v - u)$. При этом $w\in U^\perp$, $v\in (U^\perp)^\perp$,
  и $u\in U\subseteq (U^\perp)^\perp$ (мы пользуемся уже доказанным включением).
  Значит, скалярное произведение $w$ на $v-u$ равно нулю, откуда $B(w,w)=0$,
  откуда следует, что $w=0$.
  Поэтому $v = u\in U$, что и требовалось.
\end{enumerate}
\end{proof}

\begin{definition}
Пусть $(V,B)$~--- эвклидово или унитарное пространство,
$U\leq V$~--- конечномерное подпространство.
Возьмем произвольный вектор $v\in V$.
По предложению~\ref{prop:orthogonal-complement-properties-findim}
существует единственное разложение вида
$v = u + u'$, где $u\in U$, $u'\in U^\perp$.
Так определенный вектор $u\in U$ мы будем называть
\dfn{ортогональной проекцией} вектора $v$ на подпространство $U$
и обозначать через $\pr_U(v)$.
Мы получили, таким образом, отображение
$\pr_U\colon V\to V$, которое каждому вектору $v\in V$
сопоставляет его проекцию на подпространство $U$
(рассмотренную как элемент объемлющего пространства $V$).
\end{definition}

\begin{theorem}\label{thm:orth-proj-properties}
Пусть $(V,B)$~--- эвклидово или унитарное пространство,
$U\leq V$~--- конечномерное подпространство, $v\in V$.
\begin{enumerate}
\item\label{num:orth-proj-props-1}
Отображение $\pr_U\colon V\to V$ является линейным.
\item\label{num:orth-proj-props-2}
Если $v\in U$, то $\pr_U(v) = v$.
\item\label{num:orth-proj-props-3}
Если $v\in U^\perp$, то $\pr_U(v) = 0$.
\item $\Img(\pr_U) = U$.
\item $\Ker(\pr_U) = U^\perp$.
\item $v - \pr_U(v) \in U^\perp$.
\item $\pr_U\circ\pr_U = \pr_U$.
\item $||\pr_U(v)|| \leq ||v||$.
\item Если $e_1,\dots,e_n$~--- любой ортонормированный базис $U$,
то $\pr_U(v) = e_1 B(e_1,v) + \dots + e_n B(e_n,v)$.
\end{enumerate}
\end{theorem}
\begin{proof}
\begin{enumerate}
\item Пусть $v_1,v_2\in V$, причем $v_1 = u_1 + w_1$
и $v_2 = u_2 + w_2$, где $u_1,u_2\in U$, $w_1,w_2\in U^\perp$.
Тогда $v_1+v_2 = (u_1+u_2) + (w_1+w_2)$, и $u_1+u_2\in U$,
$w_1+w_2\in U^\perp$. По определению
$\pr_U(v_1) = u_1$, $\pr_U(v_2) = u_2$ и
$\pr_U(v_1+v_2) = u_1 + u_2 = \pr_U(v_1) + \pr_U(v_2)$.
Мы показали аддитивность отображения $\pr_U$. Если $v\in V$
и $v = u + w$ для $u\in U$, $w\in U^\perp$, то
$v\lambda = u\lambda + w\lambda$, откуда следует и однородность
$\pr_U$.
\item Если $v\in U$, то $v = v + 0$, где $v\in U$, $0\in U^\perp$.
\item Если $v\in U^\perp$, то $v = 0 + v$, где $0\in U$, $v\in U^\perp$.
\item В пункте (\ref{num:orth-proj-props-2}) мы показали,
что $U\subseteq\Img(\pr_U)$. Обратное включение выполнено
по определению отображения $\pr_U$.
\item В пункте (\ref{num:orth-proj-props-3}) мы показали,
что $U^\perp\subseteq\Ker(\pr_U)$. Обратно, если
$\pr_U(v) = 0$, то $v = 0 + w$, где $w\in U^\perp$.
\item По определению $v = u + w$, где $u\in U$, $w\in U^\perp$
и $u = \pr_U(v)$. Поэтому $v - \pr_U(v) = v - u = w\in U^\perp$.
\item Пусть $\pr_U(v) = u\in U$. Тогда $\pr_U(u) = u$
по пункту~(\ref{num:orth-proj-props-2}), что и требовалось.
\item $v = \pr_U(v) + w$, где $w\in U^\perp$, и потому векторы
$\pr_U(v)$ и $w$ ортогональны. По теореме Пифагора
$||v||^2 = ||\pr_U(v)||^2 + ||w||^2$, откуда следует нужное неравенство.
\item Запишем $v = u + (v-u)$,
где $u = e_1B(e_1,v) + \dots + e_n B(e_n,v)$. Как и в доказательстве
пункта~(\ref{num:orth-comp-prop-findim-1})
предложения~\ref{prop:orthogonal-complement-properties-findim},
получаем, что $v-u$ ортогонально каждому из $e_1,\dots,e_n$,
и потому $v-u\in U^\perp$, в то время как, очевидно,
$u\in U$. По определению тогда $\pr_U(v) = u$, что и требовалось.
\end{enumerate}
\end{proof}

\subsection{Сопряженные отображения}

\literature{[F], гл. XIII, \S~4, п. 2; [K2], гл. 3, \S~3, п. 1; [KM],
  ч. 2, \S~8, пп. 1--3.}

\begin{definition}
Пусть $(V,B)$ и $(V',B')$~--- эвклидовы или унитарные пространства,
$\ph\colon V\to V'$~--- линейное отображение.
Линейное отображение $\ph^*\colon V'\to V$ называется
\dfn{сопряженным}\index{сопряженное отображение} к
отображению $\ph$, если $B'(\ph(v),v') = B(v,\ph^*(v'))$ для всех
векторов $v\in V$ и $v'\in V'$.
\end{definition}

Покажем, что у каждого линейного отображения между эвклидовыми или
унитарными пространствами имеется единственное сопряженное.

\begin{proposition}
Пусть $(V,B)$ и $(V',B')$~--- эвклидовы или унитарные пространства,
$\ph\colon V\to V'$~--- линейное отображение. Существует линейное
отображение $\ph^*\colon V'\to V$ сопряженное к $\ph$. Кроме того, такое
линейное отображение единственно.
\end{proposition}

\begin{proof}
Пусть $v'\in V'$. Рассмотрим отображение $f\colon V\to k$, которое
сопоставляет вектору $v\in V$ скаляр $B'(\ph(v),v')$. Покажем, что
$f$~--- полулинейное отображение. Действительно, $f(v_1\lambda_1 +
v_2\lambda_2) = B'(\ph(v_1\lambda_1+v_2\lambda_2),v')
= B'(\ph(v_1)\lambda_1+\ph(v_2)\lambda_2,v')
= \ol{\lambda_1}B'(\ph(v_1),v') + \ol{\lambda_2}B'(\ph(v_2),v')
= \ol{\lambda_1}f(v_1) + \ol{\lambda_2}f(v_2)$.
По теореме Риса~\ref{thm:Riesz_theorem} найдется вектор
$v_f\in V$ такой, что $B(v,v_f) = f(v) = B'(\ph(v),v')$
для всех $v\in V$. Положим $\ph^*(v') = v_f$.

Таким образом, для каждого $v'\in V'$ мы нашли вектор $\ph^*(v')\in V$
такой, что $B(v,\ph^*(v')) = B'(\ph(v),v')$ для всех $v\in V$. 
Проверим, что полученное отображение $\ph^*\colon V'\to V$ является
линейным. Действительно.
\begin{align*}
B(v,\ph^*(v'_1)\lambda_1+\ph^*(v'_2)\lambda_2)
&= B(v,\ph^*(v'_1))\lambda_1 + B(v,\ph^*(v'_2))\lambda_2\\
&= B'(\ph(v),v'_1)\lambda_1 + B'(\ph(v),v'_2))\lambda_2\\
&= B'(\ph(v),v'_1\lambda_1 + v'_2\lambda_2).
\end{align*}
С другой стороны, по определению $\ph^*$ выполнено
$B(v,\ph^*(v'_1\lambda_1 + v'_2\lambda_2))
= B'(\ph(v),v'_1\lambda_1 + v'_2\lambda_2)$.
Поэтому $B(v,\ph^*(v'_1\lambda_1+v'_2\lambda_2)) =
B(v,\ph^*(v'_1)\lambda_1 -
\ph^*(v'_2)\lambda_2)$ для всех $v\in V$, откуда следует, что
$\ph^*(v'_1\lambda_1+v'_2\lambda_2) = \ph^*(v'_1)\lambda_1 -
\ph^*(v'_2)\lambda_2$.

Осталось показать единственность отображения $\ph^*$ с указанным
свойством. Но если $\tld{\ph^*}$~--- другое такое отображение, то
$B(v,\ph^*(v')) = B'(\ph(v),v') = B(v,\tld{\ph^*}(v'))$
для всех $v\in V$, $v'\in V'$.
Из этого следует, что $\ph^*(v') =
\tld{\ph^*}(v')$ для каждого $v'$.
\end{proof}

\begin{proposition}
Пусть $(V,B)$ и $(V',B')$~--- эвклидовы или унитарные пространства,
$\ph,\psi\colon V\to V'$~--- линейные отображения,
$\lambda\in k$. Тогда
\begin{enumerate}
\item $(\ph+\psi)^* = \ph^*+\psi^*$;
\item $(\lambda\ph)^* = \ol\lambda\ph^*$;
\item $(\ph^*)^* = \ph$;
\item $(\id_V)^* = \id_V$;
\item если $\eta\colon V'\to V''$~--- еще одно линейное отображение
(где $(V'',B'')$~--- эвклидово или унитарное пространство), то
$(\eta\circ\ph)^* = \ph^*\circ\eta^*$
\end{enumerate}
\end{proposition}
\begin{proof}
\begin{enumerate}
\item Пусть $v\in V$, $v'\in V'$. Тогда
\begin{align*}
B(v,(\ph+\psi)^*(v')) &= B'((\ph+\psi)(v),v') \\
&= B'(\ph(v) + \psi(v),v') \\
&= B'(\ph(v),v') + B'(\psi(v),v') \\
&= B(v,\ph^*(v')) + B(v,\psi^*(v')) \\
&= B(v,\ph^*(v')+\psi^*(v')),
\end{align*}
откуда следует, что $(\ph+\psi)^*(v') = \ph^*(v') + \psi^*(v')$,
что и требовалось.
\item Пусть $v\in V$, $v'\in V'$. Тогда
$$
B(v,(\lambda\ph)^*(v')) = B'(\lambda\ph(v),v') =
\ol\lambda B'(\ph(v),v') = \ol\lambda B(v,\ph^*(v')) = 
B(v,\ol\lambda\ph^*(v')),
$$
откуда $(\lambda\ph)^*(v') = \ol\lambda\ph^*(v')$, что и требовалось.
\item Пусть $v\in V$, $v'\in V'$. Тогда
$$
B'(v',((\ph^*)^*(v)) = B(\ph^*(v'),v) = \ol{B(v,\ph^*(v'))}
=\ol{B'(\ph(v),v')} = B'(v',\ph(v)),
$$
откуда $((\ph^*)^*(v) = \ph(v)$, что и требовалось.
\item Пусть $v,w\in V$. Тогда
$$
B(v,(\id_V)^*(w)) = B(\id_V(v),w) = B(v,w) = B(v,\id_V(w)),
$$
откуда $(\id_V)^*(w) = \id_V(w)$, что и требовалось.
\item Пусть $v\in V$, $v''\in V''$. Тогда
\begin{align*}
B(v,(\eta\circ\ph)^*(v'')) &= B''((\eta\circ\ph)(v),v'') \\
&= B''(\eta(\ph(v)),v'') \\
&= B'(\ph(v),\eta^*(v'')) \\
&= B(v,\ph^*(\eta^*(v''))) \\
&= B(v,(\ph^*\circ\eta^*)(v'')),
\end{align*}
откуда $(\eta\circ\ph)^*(v'') = (\ph^*\circ\eta^*)(v'')$,
что и требовалось.
\end{enumerate}
\end{proof}

Выясним, как выглядит матрица сопряженного отображения в
ортонормированных базисах.

\begin{proposition}\label{prop:adjoint_matrix}
Пусть $(V,B)$, $(V',B')$~--- эвклидовы или унитарные пространства,
$\mc E$~--- ортонормированный базис пространства $V$, $\mc E'$~---
ортонормированный базис пространства $V'$.
Для любого линейного отображения $\ph\colon V\to V'$ выполнено
$[\ph^*]_{\mc E',\mc E} = \ol{[\ph]_{\mc E,\mc E'}}^T$.
\end{proposition}
\begin{proof}
Обозначим $A=[\ph]_{\mc E,\mc E'}$, $A^*=[\ph^*]_{\mc E',\mc E}$.
По основному свойству матрицы линейного отображения
(теорема~\ref{thm:matrix-multiplied-by-vector}) для любых векторов
$v\in V$, $v'\in V'$ выполнено 
$A\cdot [v]_{\mc E} = [\ph(v)]_{\mc E'}$
и $A^*\cdot [v']_{\mc E'} = [\ph^*(v')]_{\mc E}$.
Матрицы Грама форм $B$ и $B'$ единичны, поэтому
$$
\ol{[\ph(v)]_{\mc E'}}^T\cdot [v']_{\mc E'} = B'(\ph(v),v') =
B(v,\ph^*(v')) =
\ol{[v]_{\mc E}}^T\cdot [\ph^*(v')]_{\mc E}.
$$
Подставляя сюда выражения для столбцов координат $\ph(v)$ и
$\ph^*(v')$, получаем
$$
\ol{A\cdot[v]_{\mc E}}^T\cdot [v']_{\mc E'} = \ol{[v]_{\mc E}}^T\cdot
A^*\cdot [v']_{\mc E'},
$$
откуда
$$
\ol{[v]_{\mc E}}^T\cdot\ol{A}^T\cdot [v']_{\mc E'} = \ol{[v]_{\mc E}}^T\cdot
A^*\cdot [v']_{\mc E'}.
$$
Это равенство верно для всех $v\in V$, $v'\in V'$. Пусть теперь $v$
пробегает все векторы базиса $\mc E$, а $v'$ пробегает все векторы
базиса $\mc E'$. Получаем равенство матриц
$A^* = \ol{A}^T$.
\end{proof}

\subsection{Самосопряженные операторы}

\begin{definition}
Пусть $(V,B)$~--- эвклидово или унитарное пространство.
Линейный оператор $T\colon V\to V$ называется \dfn{самосопряженным},
если $T^* = T$. Иными словами, $T$ самосопряжен, если
$B(T(v),w) = B(v,T(w))$ для всех $v,w\in V$.
\end{definition}

\begin{proposition}
Все собственные числа самосопряженного оператора вещественны.
\end{proposition}
\begin{proof}
Пусть $T\colon V\to V$~--- самосопряженный оператор,
$\lambda\in k$~--- собственное число оператора $T$,
и $v\in V$~--- соответствующий ему собственный вектор,
то есть, $T(v) = v\lambda$ и $v\neq 0$.
Тогда
$$
\lambda ||v||^2 = \lambda B(v,v) = B(v,v\lambda)
= B(v,T^*(v)) = B(T(v),v) = B(v\lambda,v) = \ol\lambda B(v,v)
= \ol\lambda ||v||^2
$$
При этом $||v||^2\neq 0$, и потому $\lambda=\ol\lambda$.
\end{proof}

Следующие две леммы верны только для унитарных пространств,
но не для эвклидовых
(см. замечание~\ref{rem:complex-unitary-counterexample}).

\begin{lemma}\label{lem:complex-unitary-1}
Пусть $V$~--- унитарное пространство (внимание!),
$T\colon V\to V$~--- линейный оператор.
Предположим, что $B(T(v),v) = 0$ для всех $v\in V$.
Тогда $T = 0$.
\end{lemma}
\begin{proof}
Пусть $u,v\in V$.
Заметим, что
$$
B(T(u),v) =
\frac{B(T(u+v),u+v) - B(T(u-v),u-v) - iB(T(u+vi),u+vi) + iB(T(u-vi),u-vi)}{4}
$$
(это можно проверить прямым вычислением).
В правой части стоят выражения вида $B(T(w),w)$, которые
по предположению равны нулю. Значит, $B(T(u),v)=0$.
В частности, это так для $v = T(u)$; получаем, что $T(u)=0$
для всех $u\in V$, откуда $T=0$.
\end{proof}

\begin{remark}\label{rem:complex-unitary-counterexample}
Заметим, что лемма~\ref{lem:complex-unitary-1} неверна для
эвклидовых пространств: линейный оператор $\mb R^2\to\mb R^2$,
осуществляющий поворот на $\pi/2$, служит контрпримером.
\end{remark}

\begin{lemma}
Пусть $V$~--- унитарное пространство (внимание!),
$T\colon V\to V$~--- линейный оператор.
Оператор $T$ самосопряжен тогда и только тогда, когда
скалярное произведение $B(T(v),v)$ вещественно
для всех $v\in V$.
\end{lemma}
\begin{proof}
Пусть $v\in V$.
Тогда 
$$
B(T(v),v) - \ol{B(T(v),v)} = B(T(v),v) - B(v,T(v))
= B(T(v),v) - B(T^*(v),v)
= B((T-T^*)(v),v).
$$
Если $B(T(v),v)\in\mb R$ для всех $v\in V$, то правая часть
всегда равна нулю, и по лемме~\ref{lem:complex-unitary-1}
из этого следует, что $T-T^*=0$.

Обратно, если $T = T^*$, то правая часть всегда равна нулю,
и потому $B(T(v),v) = \ol{B(T(v),v)}$ для всех $v\in V$,
откуда $B(T(v),v)\in\mb R$.
\end{proof}

\begin{remark}
Замечание~\ref{rem:complex-unitary-counterexample} показывает,
что на эвклидовом пространстве ненулевой оператор $T$ может удовлетворять
тождеству $B(T(v),v)=0$ для всех $v\in V$. Однако,
этого не может случиться для самосопряженного оператора.
\end{remark}

\begin{lemma}\label{lem:selfadjoint-zero-characterisation}
Пусть $(V,B)$~--- эвклидово или унитарное пространство,
$T\colon V\to V$~--- самосопряженный оператор.
Если $B(T(v),v) = 0$ для всех $v\in V$, то $T=0$.
\end{lemma}
\begin{proof}
Для унитарного пространства это уже доказано
в лемме~\ref{lem:complex-unitary-1}. Если же $V$ эвклидово, то
$$
B(T(u),v) = \frac{B(T(u+v),u+v) - B(T(u-v),u-v)}{4}
$$
для всех $u,v\in V$,
что проверяется прямым вычислением с использованием
равенств $B(T(v),u) = B(v,T(u)) = B(T(u),v)$
(здесь мы используем самосопряженность $T$).
По предположению правая часть равна нулю, поэтому
$B(T(u),v)=0$ для всех $u,v\in V$; в частности, это так
для $v = T(u)$, откуда следует, что $T=0$.
\end{proof}

\subsection{Нормальные операторы}

\literature{[F], гл. XIII, \S~4, п. 3; [K2], гл. 3, \S~3, п. 7; [KM],
  ч. 2, \S~8, п. 11.}

\begin{definition}
Пусть $(V,B)$~--- эвклидово или унитарное пространство.
Линейный оператор $T\colon V\to V$ называется
\dfn{нормальным}\index{оператор!нормальный}, если он коммутирует со
своим сопряженным: $T^*\circ T = T\circ T^*$.
\end{definition}

\begin{remark}
Очевидно, что любой самосопряженный оператор нормален.
\end{remark}

\begin{lemma}[Свойства нормальных операторов]
\begin{enumerate}
\item Тождественный оператор нормален.
\item Сопряженный к нормальному оператору нормален.
\end{enumerate}
\end{lemma} 
\begin{proof}
Очевидно.
\end{proof}

\begin{lemma}\label{prop:normal-operator-equiv}
Пусть $(V,B)$~--- эвклидово или унитарное пространство.
Оператор $T\colon V\to V$ нормален тогда и только тогда, когда
$||T(v)|| = ||T^*(v)||$ для всех $v\in V$.
\end{lemma}
\begin{proof}
Заметим, что оператор $T^*\circ T - T\circ T^*$ самосопряжен.
По лемме~\ref{lem:selfadjoint-zero-characterisation}
равенство $T^*\circ T - T\circ T^*$ нулю равносильно тому,
что $B((T^*\circ T - T\circ T^*)(v),v) = 0$ для всех $v\in V$,
что равносильно равенству
$B(T^*(T(v)),v) = B(T(T^*(v)),v)$ для всех $v\in V$.
Но $B(T^*(T(v)),v) = ||T(v)||^2$ и $B(T(T^*(v)),v) = ||T^*(v)||^2$.
\end{proof}

\begin{proposition}\label{prop:normal-operator-adjoint-eigenvalues}
Пусть $(V,B)$~--- эвклидово или унитарное пространство,
$T\colon V\to V$~--- нормальный оператор, и $v\in V$~--- собственный
вектор оператора $T$, соответствующий собственному числу $\lambda$.
Тогда $v$ является и собственным вектором оператора $T^*$,
соответствующим собственному числу $\ol\lambda$.
\end{proposition}
\begin{proof}
Из нормальности $T$ следует, что и оператор $T - \lambda\id_V$
нормален (проверьте это!).
По лемме~\ref{prop:normal-operator-equiv} тогда
$||(T-\lambda\id_V)(v)|| = ||(T-\lambda\id_V)^*(v)||$.
Но левая часть по предположению равна нулю,
а правая часть равна $||(T^*-\ol\lambda\id_V)(v)||$.
\end{proof}

\begin{proposition}
Пусть $(V,B)$~--- эвклидово или унитарное пространство,
$T\colon V\to V$~--- нормальный оператор. Тогда собственные векторы
$T$, соответствующие различным собственным числам, ортогональны.
\end{proposition}
\begin{proof}
Пусть $\lambda\neq\mu$~--- два различных собственных числа
оператора $T$, и пусть $u,v\in V$~--- соответствующие им
собственные векторы: $T(u) = u\lambda$, $T(v) = v\mu$.
По предложению~\ref{prop:normal-operator-adjoint-eigenvalues}
теперь $T^*(u) = u\ol\lambda$.
Поэтому $(\lambda-\mu)B(u,v) = B(u\ol\lambda,v) - B(u,v\mu)
= B(T^*(u),v) - B(u,T(v)) = 0$.
Поскольку $\lambda\neq\mu$, из этого равенства следует, что
$B(u,v)=0$, что и требовалось.
\end{proof}

\subsection{Спектральные теоремы}

\literature{[F], гл. XIII, \S~5; [K2], гл. 3, \S~3, пп. 3, 6; [KM],
  ч. 2, \S~7, пп. 4--5; \S~8, пп. 2--6, 8.}

\begin{theorem}[Спектральная теорема для нормальных операторов в
унитарном пространстве]\label{thm:spectral-unitary}
Пусть $(V,B)$~--- унитарное пространство,
$T\colon V\to V$~--- линейный оператор.
Следующие условия равносильны:
\begin{enumerate}
\item оператор $T$ нормален;
\item у $V$ есть ортонормированный базис, состоящий из собственных
векторов оператора $T$;
\item матрица оператора $T$ в некотором ортонормированном базисе
$V$ диагональна.
\end{enumerate}
\end{theorem}
\begin{proof}
Очевидно, что $(2)\Leftrightarrow(3)$ (см. также
доказательство теоремы~\ref{thm:diagonalizable-equivalent}).
Покажем, что из (3) следует (1). Пусть матрица $T$ в некотором
ортонормированном базисе $\mc B$ диагональна.
По предложению~\ref{prop:adjoint_matrix}
матрица $T^*$ тогда получается из матрицы $T$ транспонированием
и сопряжением, и потому тоже диагональна. Но любые две диагональные
матрицы коммутируют; поэтому $T$ коммутирует с $T^*$,
то есть, $T$ нормален.

Пусть теперь выполняется (1): оператор $T$ нормален.
По теореме о жордановой форме~\ref{thm:jordan-form} существует
базис $\mc B = (v_1,\dots,v_n)$ пространства $V$, в котором матрица $T$
верхнетреугольна. Применим к этому базису процесс ортогонализации
Грама--Шмидта: мы получим ортонормированный базис 
$\mc E = (e_1,\dots,e_n)$.
По предложению~\ref{prop:ut-equivalent-defs} верхнетреугольность
матрицы $T$ в базисе $\mc B$ равносильна тому, что
все подпространства вида $\la v_1,\dots,v_i\ra$ являются
$T$-инвариантными. Но в процессе ортогонализации
мы получили базис, для которого
$\la e_1,\dots,e_i\ra = \la v_1,\dots,v_i\ra$,
а инвариантность этих подпространств равносильна
верхнетреугольности матрицы $T$ в ортонормированном базисе $\mc E$.

Итак, матрица оператора $T$ в базисе $\mc E$ верхнетреугольна:
$$
[T]_{\mc E} = \begin{pmatrix}
a_{11} & a_{12} & \dots & a_{1n} \\
0 & a_{22} & \dots & a_{2n} \\
\vdots & \vdots & \ddots & \vdots \\
0 & 0 & \dots & a_{nn}
\end{pmatrix}
$$
Покажем, что она на самом деле
не только верхнетреугольна, но и диагональна.
Мы знаем, что матрица оператора $T^*$ в том же базисе выглядит так:
$$
[T^*]_{\mc E} = \overline{[T]_{\mc E}}^T\begin{pmatrix}
\ol{a_{11}} & 0 & \dots & 0 \\
\ol{a_{12}} & \ol{a_{22}} & \dots & 0 \\
\vdots & \vdots & \ddots & \vdots \\
\ol{a_{1n}} & \ol{a_{2n}} & \dots & \ol{a_{nn}}
\end{pmatrix}
$$
Самое время воспользоваться нормальностью оператора $T$.
Посмотрим внимательно, что стоит в левом верхнем углу матриц,
полученных перемножением $[T]_{\mc E}$ и $[T^*]_{\mc E}$.
Нетрудно видеть, что у матрицы $[T^*]\cdot [T]$ в позиции $(1,1)$
стоит $|a_{11}|^2$, а у матрицы $[T]\cdot [T^*]$~---
$|a_{11}|^2 + |a_{12}|^2 + \dots + |a_{1n}|^2$,
сумма квадратов модулей элементов первой строки матрицы $[T]$.
Но эти выражения должны быть равны, и все входящие в них слагаемые~---
неотрицательные вещественные числа. Поэтому
$a_{12} = \dots = a_{1n} = 0$. Значит, в первой строке матрицы $[T]$
на самом деле только один ненулевой элемент: диагональны.
Вооружившись этим знанием, проследим теперь за позицией $(2,2)$.
Перемножая матрицы в одном порядке, получаем $|a_{22}|^2$,
а в другом~--- сумму квадратов элементов второй строки матрицы $[T]$.
Из этого следует, что и во второй строке матрица $[T]$ не отличается
от диагональной. Продолжая этот процесс, получаем,
что $[T]_{\mc E}$ диагональна, что и требовалось.
\end{proof}

Теперь обратимся к случаю эвклидового пространства. Как мы знаем,
жорданова форма для оператора на вещественном пространстве уже не
обязана быть верхнетреугольной, поэтому для переноса спектральной
теоремы на эвклидов случай придется действовать обходным путем.
Сначала мы разберемся с самосопряженными операторами.
Для этого нам понадобится следующая лемма, в основе которой лежит
несложное вычисление, известное вам со школы:
$$
x^2 + bx + c = \left(x+\frac{b}{2}\right)^2 +
\left(c-\frac{b^2}{4}\right).
$$

\begin{lemma}\label{lem:quadratic-operator-invertible}
Пусть $T\colon V\to V$~--- самосопряженный линейный оператор
на эвклидовом или унитарном пространстве $V$,
и числа $b,c\in\mb R$ таковы, что $b^2-4c<0$.
Тогда оператор $T^2 + bT + c\id_V$ обратим.
\end{lemma}
\begin{proof}
Пусть $v\in V$. Тогда
\begin{align*}
B((T^2 + bT + c\id_V)(v),v) &= B(T^2(v),v) + bB(T(v),v) + cB(v,v) \\
&= B(T(v),T(v)) + bB(T(v),v) + c||v||^2 \\
&\geq ||T(v)||^2 - |b|\cdot ||T(v)||\cdot ||v|| + c||v||^2
\end{align*}
в силу неравенства Коши--Буняковского--Шварца:
$-||T(v)||\cdot ||v|| \leq B(T(v),v) \leq ||T(v)||\cdot ||v||$.
Полученное выражение можно переписать так:
$$
\left(||T(v)|| - \frac{|b|\cdot ||v||}{2}\right)^2 +
\left(c-\frac{b^2}{4}\right)||v||^2,
$$
и видно, что оно (при нашем условии на $b$ и $c$) неотрицательно.
Поэтому оператор $T^2 + bT + c\id$ инъективен, значит, и биективен.
\end{proof}

\begin{remark}
Мы знаем, что у любого оператора на комплексном пространстве есть
собственное число.
Поэтому следующую лемму достаточно доказать только для случая
эвклидово пространств.
\end{remark}

\begin{lemma}\label{lem:real-self-adjoint-has-eigenvalue}
Пусть $V \neq \{0\}$~--- эвклидово пространство, $T\colon V\to V$~---
самосопряженный линейный оператор. Тогда у $T$ есть собственное
число.
\end{lemma}
\begin{proof}
Пусть $\dim(V) = n$. Рассмотрим минимальный многочлен оператора $T$:
$$
f = a_0 + a_1x + \dots + a_nx^n \in k[x]
$$
(см. определение~\ref{dfn:minimal-polynomial}).
По теореме~\ref{thm_irreducible_real} его можно разложить на множители
вида
$$
f = c(x^2 + b_1x + c_1)\dots (x^2 + b_Mx c_M)
(x-\lambda_1)\dots(x-\lambda_m),
$$
где $c\neq 0$, $b_j,c_j,\lambda_j$~--- вещественные числа, причем
$b_j^2 - 4c_j < 0$. Поэтому
$$
0 = f(T)(v) = c(T^2 + b_1T + c_1\id)\dots(T^2+b_MT+c_M\id)
(T-\lambda_1\id)\dots(T-\lambda_m\id)(v).
$$
По лемме~\ref{lem:quadratic-operator-invertible} множители вида
$T^2 + b_jT + c_j\id$ обратимы. Поэтому
$$
0 = (T-\lambda_1\id)\dots (T-\lambda_m\id)(v).
$$
Значит, хотя бы один из операторов $T-\lambda_j\id$ неинъективен.
Это и означает, что у $T$ есть собственное число.
\end{proof}

\begin{remark}
Позже мы увидим (см.~\ref{prop:normal-operator-invariant-subspaces}),
что в следующем предложении можно
заменить условие самосопряженности оператора на условие нормальности.
\end{remark}

\begin{proposition}\label{prop:orthogonal-complement-invariant}
Пусть $T\colon V\to V$~--- самосопряженный оператор на эвклидовом или
унитарном пространстве, и пусть $U\leq V$~--- $T$-инвариантное
подпространство.
Тогда
\begin{enumerate}
\item подпространство $U^\perp$ также $T$-инвариантно;
\item оператор $T|_U$ самосопряжен;
\item оператор $T|_{U^\perp}$ самосопряжен.
\end{enumerate}
\end{proposition}
\begin{proof}
\begin{enumerate}
\item 
Пусть $v\in U^\perp$. Нам хочется показать, что $T(v)\in U^\perp$.
Возьмем любой вектор $u\in U$ и посмотрим на $B(T(v),u)$.
Из самосопряженности $T$ следует,
что $B(T(v),u) = B(v,T(u))$. Но по условию $T(u)\in U$, значит,
мы получили $0$.
\item Если $u,v\in U$, то $B((T|_U)(u),v) = B(T(u),v) = B(u,T(v))
= B(u,(T|_U)(v))$.
\item Применим результат второго пункта к $U^\perp$ вместо $U$.
\end{enumerate}
\end{proof}

\begin{theorem}[Спектральная теорема для самосопряженных операторов в
эвклидовых пространствах]\label{thm:spectral-real-self-adjoint}
Пусть $(V,B)$~--- эвклидово пространство,
$T\colon V\to V$~--- линейный оператор.
Следующие условия равносильны:
\begin{enumerate}
\item оператор $T$ самосопряжен;
\item у $V$ есть ортонормированный базис, состоящий из собственных
векторов оператора $T$;
\item матрица оператора $T$ в некотором ортонормированном базисе
$V$ диагональна.
\end{enumerate}
\end{theorem}
\begin{proof}
Мы уже знаем, что $(2)\Leftrightarrow (3)$. Предположим, что
выполняется $(3)$: матрица оператора $T$ в некотором базисе
диагональна. Но диагональная матрица совпадает со своей
транспонированной, поэтому $T=T^*$, откуда следует $(1)$.

Теперь мы докажем, что из $(1)$ следует $(2)$ индукцией по размерности
пространства $V$.
Если $\dim(V)=1$, утверждение очевидно.
Пусть теперь $\dim(V) > 1$, и оператора $T$ самосопряжен.
По лемме~\ref{lem:real-self-adjoint-has-eigenvalue} у $T$ есть
собственное число и, стало быть, собственный вектор $u$.
Поделив его на $||u||$, можно считать, что $||u|| = 1$.
Подпространство $U = \la u\ra$ тогда является $T$-инвариантным, и по
предложению~\ref{prop:orthogonal-complement-invariant}
подпространство $U^\perp$ тоже $T$-инвариантно,
и оператор $T|_{U^\perp}$ самосопряжен.
По предположению индукции у $U^\perp$ есть ортонормальный базис,
состоящий из собственных векторов оператора $T|_{U^\perp}$.
Присоединив к нему $u$, получаем ортонормальный базис $V$,
состоящий из собственных векторов оператора $T$.
\end{proof}

Теперь мы готовы описать нормальные операторы на двумерных эвклидовых
пространствах.

\begin{proposition}\label{prop:real-normal-not-self-adjoint-dim-2}
Пусть $V$~--- эвклидово пространство размерности $2$,
$T\colon V\to V$~--- линейный оператор.
Следующие условия равносильны:
\begin{enumerate}
\item $T$ нормален, но не самосопряжен;
\item матрица $T$ в любом ортонормальном базисе $V$ имеет вид
$$
\begin{pmatrix} \alpha & -\beta \\ \beta & \alpha\end{pmatrix},
$$
где $\beta\neq 0$;
\item матрица $T$ в некотором ортонормальном базисе $V$ имеет вид
$$
\begin{pmatrix} \alpha & -\beta \\ \beta & \alpha\end{pmatrix},
$$
где $\beta > 0$.
\end{enumerate}
\end{proposition}
\begin{proof}
$(1)\Rightarrow (2)$. Пусть $e_1,e_2$~--- ортонормальный базис
пространства $V$, и пусть матрица $T$ в этом базисе имеет вид
$$
\begin{pmatrix}\alpha & \gamma\\\beta & \delta\end{pmatrix}.
$$
Тогда $||T(e_1)||^2 = \alpha^2 + \beta^2$, $||T^*(e_1)||^2 = \alpha^2 + \gamma^2$.
По предложению~\ref{prop:normal-operator-equiv} эти числа равны,
откуда $\gamma = \pm \beta$. Если $\gamma=\beta$, то $T$ самосопряжен (его матрица
симметрична), поэтому $\gamma = -\beta$, при этом $\beta\neq 0$.
Перемножим теперь матрицы
$T$ и $T^*= T^T$ в одном и в другом порядке. Результаты должны
совпасть, но в правом верхнем углу у одной матрицы стоит $\beta\delta$, а у
другой $\alpha\beta$. Значит, $\alpha=\delta$, и мы получили матрицу нужного вида.

$(2)\Rightarrow (3)$. Если в нашем базисе уже $\beta>0$, то все доказано,
а если нет~--- поменяем знак у второго базисного вектора.

$(3)\Rightarrow (1)$. Если $T$ имеет указанный вид, то видно, что $T$
не самосопряжен. Перемножая матрицы $T$ и $T^*$ видим, что $T$
нормален.
\end{proof}

\begin{proposition}\label{prop:normal-operator-invariant-subspaces}
Пусть $(V,B)$~--- эвклидово или унитарное пространство,
$T\colon V\to V$~--- нормальный оператор, $U\leq V$~---
$T$-инвариантное подпространство. Тогда
\begin{enumerate}
\item подпространство $U^\perp$ тоже $T$-инвариантно;
\item подпространство $U$ $T^*$-инвариантно;
\item $(T|_U)^* = (T^*)|_U$;
\item операторы $T|_U$ и $T|_{U^\perp}$ нормальны.
\end{enumerate}
\end{proposition}
\begin{proof}
Пусть $e_1,\dots,e_m$~--- какой-нибудь ортонормированный базис
$U$. Дополним его до ортонормированного базиса $\mc B$ пространства
$V$ векторами $f_1,\dots,f_n$. Матрица оператора $T$ имеет в этом
базисе следующий вид:
$$
[T]_{\mc B} = \begin{pmatrix} A & B \\ 0 & C\end{pmatrix},
$$
где $A$~--- блок размера $m\times m$, а $C$~--- блок размера
$n\times n$.
Нетрудно понять, что $||T(e_j)||^2$ равняется сумме квадратов модулей
элементов $j$-го столбца матрицы $A$. Складывая по всем $j$,
получаем, что $\sum_j||T(e_j)||^2$ равна сумме квадратов модулей всех
элементов матрицы $A$.
С другой стороны, $||T^*(e_j)||^2$ равна сумме квадратов модулей
элементов $j$-й строки матрицы $A$ и $j$-й строки матрицы $B$.
Складывая по всем $j$, получаем, что $\sum_j||T^*(e_j)||^2$ равна
сумме квадратов модулей всех элементов матрицы $A$ и всех элементов
матрицы $B$.
Из равенства $||T(e_j)|| = ||T^*(e_j)||$
(предложение~\ref{prop:normal-operator-equiv}) теперь следует,
что $B$~--- нулевая матрица. Теперь из вида матрицы оператора $T$
можно заключить, что $U^\perp$ $T$-инвариантно. Написав матрицу
оператора $T^*$, можно заметить, что $U$ еще и $T^*$-инвариантно.

Докажем $(3)$. Пусть $S = T|_U\colon U\to U$. Возьмем $v\in U$.
Тогда $B(u,S^*(v)) = B(S(u),v) = B(T(u),v) = B(u,T^*(v)$ для всех
$u\in U$. Мы уже знаем, что $T^*(v)\in U$, поэтому из приведенного
равенства следует, что $S^*(v) = T^*(v)$.
Это выполнено для всех $v\in U$, потому
$(T|_U)^* = (T^*)|_U$.

Наконец, для доказательства $(4)$ можно заметить, что $T$ коммутирует
с $T^*$, и потому $T|_U$ коммутирует с $(T|_U)^* = (T^*)|_U$;
подставляя $U^\perp$ вместо $U$, видим, что и
$T|_{U^\perp}$ нормален.
\end{proof}

\begin{theorem}[Спектральная теорема для нормальных операторов в
эвклидовом пространстве]\label{thm:spectral-euclidean}
Пусть $(V,B)$~--- эвклидово пространство, и пусть $T\colon V\to V$~---
линейный оператор.
Следующие условия равносильны:
\begin{enumerate}
\item оператор $T$ нормален;
\item существует ортонормированный базис пространства $V$, в котором
матрица оператора $T$ блочно-диагональна, причем каждый блок имеет
либо размер $1\times 1$, либо размер $2\times 2$ и вид
$$
\begin{pmatrix} \alpha & -\beta \\ \beta & \alpha\end{pmatrix},
$$
где $\beta > 0$.
\end{enumerate}
\end{theorem}
\begin{proof}
$(2)\Rightarrow (1)$: несложно проверить, что матрица такого вида
коммутирует со своей сопряженной.

Докажем $(1)\Rightarrow (2)$ индукцией по размерности $V$.
Случай $\dim(V)=1$ тривиален, а случай $\dim(V) = 2$ следует из
спектральной теоремы~\ref{thm:spectral-real-self-adjoint} для
самосопряженного оператора, и из
предложения~\ref{prop:real-normal-not-self-adjoint-dim-2}
для остальных.

Пусть теперь $\dim(V) > 2$.
Если у оператора $T$ есть одномерное инвариантное подпространство
(иными словами, есть собственное число), обозначим его через $U$.
Если же нет, то 
по предложению~\ref{prop:real-operator-invariant-subspace} у него
есть двумерное инвариантное подпространство, и тогда мы обозначим его
через $U$.
Если $\dim(U) = 1$, выберем в $U$ вектор нормы $1$~--- это будет
ортонормированным базисом подпространства $U$; если же $\dim(U) = 2$,
то оператор $T|_U$ нормален
(по предложению~\ref{prop:normal-operator-invariant-subspaces}), но не
самосопряжен (иначе у $T|_U$ было бы собственное число
по лемме~\ref{lem:real-self-adjoint-has-eigenvalue}), и в этом случае
можно применить
предложение~\ref{prop:real-normal-not-self-adjoint-dim-2}.

В любом случае, мы нашли ортонормированный базис в инвариантном
подпространстве $U$, причем подпространство $U^\perp$ $T$-инвариантно,
и оператор $T|_{U^\perp}$ нормален
(по предожению~\ref{prop:normal-operator-invariant-subspaces}).
По предположению индукции у $U^\perp$ есть ортонормированный базис с
нужными свойствами; приписывая к нему выбранный базис $U$,
получаем нужный базис всего пространства $V$.
\end{proof}


\subsection{Самосопряженные, кососимметрические, унитарные,
  ортогональные операторы}

\literature{[F], гл. XIII, \S~5; [K2], гл. 3, \S~3, пп. 3, 6; [KM],
  ч. 2, \S~7, пп. 1--2, 4; \S~8, пп. 2--6.}
\nopagebreak

Сейчас мы применим знания, полученные при изучении нормальных
операторов, к некоторым частным случаям.

\begin{definition}
Пусть $(V,B)$~--- эвклидово или унитарное пространство,
$a\colon V\to V$~--- линейный оператор.
Оператор $a$ называется
\dfn{самосопряженным}\index{оператор!самосопряженный}, если он
совпадает со своим сопряженным: $a = a^*$. Оператор $a$ называется
\dfn{кососимметрическим}\index{оператор!кососимметрический}, если он
противоположен своему сопряженному:
$a = -a^*$. Если выполняется равенство $a\circ a^* = a^*\circ a =
\id_V$, то оператор $a$ называется
\dfn{унитарным}\index{оператор!унитарный} в случае унитарного
пространства и \dfn{ортогональным}\index{оператор!ортогональный} в
случае эвклидового пространства.
\end{definition}

\begin{remark}
Нетрудно видеть, что самосопряженные, кососимметрические, унитарные,
ортогональные операторы являются нормальными.
\end{remark}

\begin{theorem}\label{thm:unitary_canonical_forms}
Пусть $(V,B)$~--- конечномерное унитарное пространство,
$a\colon V\to V$~--- линейный оператор.
\begin{enumerate}
\item Оператор $a$ является самосопряженным тогда и
только тогда, когда существует ортонормированный базис пространства
$V$, в котором матрица оператора $a$ диагональна, и все ее
диагональные элементы вещественны.
\item Оператор $a$ является кососимметрическим тогда и
только тогда, когда существует ортонормированный базис пространства
$V$, в котором матрица оператора $a$ диагональна, и все ее
диагональные элементы~--- чисто мнимые комплексные числа.
\item Оператор $a$ является унитарным тогда и
только тогда, когда существует ортонормированный базис пространства
$V$, в котором матрица оператора $a$ диагональна, и все ее
диагональные элементы~--- комплексные числа, равные по модулю $1$.
\end{enumerate}
\end{theorem}
\begin{proof}
Если оператор самосопряженный, кососимметрический, унитарный, то по
теореме~\ref{thm:spectral-unitary} существует базис, в котором его
матрица диагональна. Если он самосопряжен, то каждый диагональный
блок $1\times 1$ самосопряжен, поэтому в нем стоит комплексное число
$\lambda$ такое, что $\lambda=\ol\lambda$, то есть, $\lambda\in\mb R$.
Аналогично, из кососимметричности следует, что $\lambda$ чисто мнимое,
а из унитарности~--- то, что $|\lambda|^2 = \lambda\ol\lambda = 1$.

Обратно, если все диагональные элементы матрицы имеют указанный вид,
то прямая проверка показывает, что оператор $a$ обладает
соответствующим свойством.
\end{proof}

\begin{theorem}\label{thm:euclidean_canonical_forms}
Пусть $(V,B)$~--- конечномерное эвклидово пространство,
$a\colon V\to V$~--- линейный оператор.
\begin{enumerate}
\item Оператор $a$ является самосопряженным тогда и
только тогда, когда существует ортонормированный базис пространства
$V$, в котором матрица оператора $a$ диагональна.
\item Оператор $a$ является кососимметрическим тогда и
только тогда, когда существует ортонормированный базис пространства
$V$, в котором матрица оператора $a$ имеет блочно-диагональный
вид, и каждый блок выглядит как $(0)$ или  $\begin{pmatrix} 0 & -\beta
  \\ \beta & 0\end{pmatrix}$ для $\beta\in\mb R$, $\beta > 0$.
\item Оператор $a$ является ортогональным тогда и
только тогда, когда существует ортонормированный базис пространства
$V$, в котором матрица оператора $a$ имеет блочно-диагональный
вид, и каждый блок выглядит как $(1)$, $(-1)$
или $\begin{pmatrix}\alpha&-\beta\\ \beta & \alpha\end{pmatrix}$ для
$\alpha,\beta\in\mb R$, $\beta > 0$, $\alpha^2 + \beta^2 = 1$.
\end{enumerate}
\end{theorem}
\begin{proof}
Если оператор самосопряженный, кососимметрический, нормальный, то по
теореме~\ref{thm:spectral-euclidean} существует базис, в котором его
матрица блочно-диагональна, с блоками вида
$$
\begin{pmatrix}
\alpha & -\beta\\
\beta & \alpha
\end{pmatrix},
$$
где $b>0$.
Если он самосопряжен, то каждый диагональный блок самосопряжен, что
для блока $2\times 2$ указанного вида означает, что $\beta=-\beta$,
что невозможно. Поэтому остаются только блоки размера $1\times 1$,
что означает диагональность матрицы. Аналогично, из кососимметричности
для блока $2\times 2$ следует, что $\alpha=0$, а для блока $(\lambda)$
размера $1\times 1$~--- что $\lambda = 0$. Наконец, из ортогональности
для блока $2\times 2$ следует, что $\alpha^2+\beta^2=1$, а для блока
$(\lambda)$~--- что $\lambda^2=1$, откуда следует, что $\lambda=\pm 1$.

Обратно, если матрица оператора состоит из блоков указанного вида,
нетрудно проверить, что оператор обладает соответствующим свойством.
\end{proof}

\begin{definition}
Пусть $(V,B)$~--- эвклидово или унитарное пространство,
$a\colon V\to V$~--- линейный оператор.
Будем говорить, что оператор $a$ \dfn{сохраняет скалярное
  произведение}\index{оператор!сохраняет скалярное произведение},
если $B(a(u),a(v))=B(u,v)$ для любых $u,v\in V$.
Оператор $a$ называется \dfn{изометрией}\index{изометрия}, если
$||a(v)|| = ||v||$ для всех $v\in V$.
\end{definition}

\begin{lemma}\label{lem:isometry_equiv}
Пусть $a\colon V\to V$~--- линейный оператор на эвклидовом или
унитарном пространстве $(V,B)$. Следующие условия равносильны:
\begin{enumerate}
\item $a$ ортогонален (в случае эвклидова пространства) или унитарен
  (в случае унитарного пространства);
\item $a$ сохраняет скалярное произведение;
\item $a$ является изометрией.
\end{enumerate}
\end{lemma}
\begin{proof}
\begin{itemize}
\item[$1\Rightarrow 2$] Пусть $a$ ортогонален/унитарен. Тогда
  $B(a(u),a(v)) = B(u,a^*(a(v)))$ по определению сопряженного оператора;
  из равенства $a^*\circ a = \id$ теперь следует, что $B(a(u),a(v)) =
  B(u,v)$.
\item[$2\Rightarrow 1$] Пусть $B(a(u),a(v))= B(u,v)$ для всех $u,v\in
  V$. По определению сопряженного оператора $B(a(u),a(v)) =
  B(u,a^*(a(v)))$. Стало быть, $B(u,v) = B(u,a^*(a(v)))$ для всех
  $u,v\in V$.  Значит, вектор $v-a^*(a(v))$ ортогонален всем векторам $u\in V$,
  откуда следует, что  $v = a^*(a(v))$ для
  всех $v\in V$. Поэтому $a^*\circ a = \id$.
\item[$2\Rightarrow 3$] Если $a$ сохраняет скалярное произведение, то,
  в частности, $B(a(v),a(v)) = B(v,v)$ для всех $v\in V$. Левая часть
  равна $||a(v)||^2$, а правая равна $||v||^2$. Извлекая
  [положительные] квадратные корни, получаем, что $a$ является
  изометрией.
\item[$3\Rightarrow 2$] Если $a$ является изометрией, то
  $B(a(u+\lambda v),a(u+\lambda v)) = B(u+\lambda v,u+\lambda
  v)$. Раскроем скобки:
  \begin{align*}
  &B(a(u),a(u)) + \ol\lambda B(a(v),a(u)) + \lambda B(a(u),a(v)) +
  \ol\lambda\lambda B(a(v),a(v))\\ &= B(u,u) + \ol\lambda B(v,u) +
  \lambda B(u,v) + \ol\lambda\lambda B(v,v).
  \end{align*}
  Воспользуемся равенствами $B(a(x),a(x)) = B(x,x)$ и $B(x,y) =
  \ol{B(y,x)}$:
  $$
  \lambda B(a(u),a(v)) + \ol{\lambda B(a(u),a(v))} =
  \lambda B(u,v) + \ol{\lambda B(u,v)}.
  $$
  Подставляя $\lambda=1$ и $\lambda = i$, получаем равенства
  $$
  2\Ree(B(a(u),a(v)) = 2\Ree(B(u,v)), \quad
  2\Img(B(a(u),a(v)) = 2\Img(B(u,v)).
  $$
  Отсюда следует, что $B(a(u),a(v)) = B(u,v)$, что и требовалось.
\end{itemize}
\end{proof}

\begin{corollary}[Теорема Эйлера о вращениях трехмерного пространства]
Пусть $V = \mb R^3$~--- трехмерное вещественное пространство со
стандартным эвклидовым скалярным произведением, $a\colon\mb
R^3\to\mb R^3$~--- изометрия на $\mb R^3$. Тогда в некотором
ортонормированном базисе матрица оператора $a$ имеет вид
$$
\begin{pmatrix}
\pm 1 & 0 & 0\\
0 & \cos(\ph) & \sin(\ph)\\
0 & -\sin(\ph) & \cos(\ph)
\end{pmatrix}
$$
для некоторого угла $\ph$.
Если, кроме того, определитель оператора $a$ равен $1$, то элемент в
левом верхнем углу такой матрицы равен $1$.
\end{corollary}
\begin{proof}
По лемме~\ref{lem:isometry_equiv} оператор $a$ ортогонален. По
теореме~\ref{thm:euclidean_canonical_forms} найдется ортонормированный
базис $V$, в котором матрица оператора $a$ имеет блочно-диагональный
вид, и блоки имеют вид $(\pm 1)$ или
$\begin{pmatrix}\cos(\ph)&\sin(\ph)\\-\sin(\ph)&\cos(\ph)\end{pmatrix}$. Если
там имеется блок размера $2$, то теорема доказана. Если же все блоки
имеют размер $1$, то среди знаков $\pm 1$ найдется два одинаковых, и
их можно заменить на блок размера $2$ вида
$\begin{pmatrix}\cos(\ph)&\sin(\ph)\\-\sin(\ph)&\cos(\ph)\end{pmatrix}$
для $\ph=0$ или $\ph = \pi$. Последнее утверждение теоремы очевидно.
\end{proof}

\begin{corollary}[Приведение вещественной квадратичной формы к
  диагональному виду при помощи ортогонального преобразования]
Пусть $(V,B)$~--- эвклидово пространство, и пусть
$q\colon V\times V\to \mb R$~--- симметрическая билинейная
форма. Существует ортогональный базис пространства $V$, в котором
матрица Грама формы $q$ имеет диагональный вид.
\end{corollary}
\begin{proof}
Выберем некоторый ортонормированный базис $\mc B$ пространства $V$;
пусть $Q$~--- матрица Грама формы $q$ в этом базисе.
Поскольку форма $q$ симметрична, матрица $Q$ является симметрической
матрицей: $Q^T = Q$. Рассмотрим $Q$ как матрицу некоторого оператора
$a$ на пространстве $V$; по предложению~\ref{prop:adjoint_matrix}
оператор $a$ самосопряжен.
По теореме~\ref{thm:euclidean_canonical_forms} существует
ортонормированный базис $\mc C$ пространства $V$, в котором матрица
оператора $a$ диагональна. Это означает, что
$C^{-1}QC = D$~--- диагональная матрица, где $C$~--- матрица перехода
от базиса $\mc B$ к базису $\mc C$
(см. теорему~\ref{thm_matrix_under_change_of_bases}). Кроме того,
поскольку $C$~--- матрица перехода между ортонормированными базисами,
то $C$ ортогональна (лемма~\ref{lem:orthogonal_equivalencies}): $C^T =
C^{-1}$. Но тогда
$D = C^TQC$, и по теореме~\ref{thm:Gram_matrix_change_of_coordinates}
это означает, что $D$~--- матрица Грама
квадратичной формы $q$ в ортонормированном базисе $\mc C$.
\end{proof}

\begin{remark}\label{rem:self_adjoint_geometry}
Переформулируем утверждение первого пункта
теоремы~\ref{thm:euclidean_canonical_forms} на геометрическом языке.
Если $a$~--- самосопряженный оператор на эвклидовом пространстве $V$,
мы показали, что в некотором ортонормированном базисе его матрица $A$
имеет диагональный вид. Пусть $\lambda_1,\dots,\lambda_m$~--- все
различные собственные числа $a$; тогда у матрицы $A$ на диагонали
стоят числа $\lambda_1,\dots,\lambda_m$ (возможно, некоторые
встречаются по несколько раз). Очевидно, что собственное
подпространство, соответствующее $\lambda_i$~--- это в точности
линейная оболочка базисных векторов, соответствующих позициям, в
которых на диагонали стоит $\lambda_i$. Поскольку базис
ортонормирован, собственные подпространства, соответствующие различным
собственным числам, попарно ортогональны; кроме того, их прямая сумма
совпадает со всем пространством $V$ (см. также
раздел~\ref{subsect:diagonalizable}).

Таким образом, каждому самосопряженному оператору на $V$ мы сопоставили
разложение пространства $V$ в ортогональную прямую сумму
собственных подпространств, соответствующих различным собственным
числам этого оператора.
Обратно, если имеется разложение пространства $V$ в ортогональную
прямую сумму подпространств $V=\bigoplus_{i=1}^{m}V_m$ и заданы
различные числа $\lambda_1,\dots,\lambda_m$, то имеется единственный
самосопряженный оператор $a$, который на векторе $v=\sum_{i=1}^m v_i$ (для
$v_i\in V_i$) действует следующим образом: $a(v) = \sum_{i=1}^m
\lambda_i v_i$. Если в каждом подпространстве $V_i$ выбрать
ортонормированный базис, то объединение этих базисов является
ортонормированным базисом пространства $V$, и матрица оператора $a$ в
этом базисе диагональна; на диагонали стоят числа
$\lambda_1,\dots,\lambda_m$, и кратность $\lambda_i$ равна размерности
подпространства $V_i$.

Мы получили взаимно однозначное соответствие между самосопряженными
операторами и разложениями $V=\bigoplus_{i=1}^m V_i$ с заданными
попарно различными числами $\lambda_1,\dots,\lambda_m$.
\end{remark}

\subsection{Положительно определенные операторы}

\literature{[F], гл. XIII, \S~4, п. 4; [K2], гл. 3, \S~3, пп. 8, 9.}
\nopagebreak

Пусть $(V,B)$~--- эвклидово или унитарное пространство, $a\colon V\to
V$~--- самосопряженный оператор на нем.
Тогда в силу самосопряженности $B(a(v),v) = B(v,a(v))$ для любого $v\in
V$; с другой стороны, $B(a(v),v) = \overline{B(v,a(v))}$. Поэтому
выражение $B(a(v),v)$ всегда вещественно.

\begin{definition}
Самосопряженный оператор $a\colon V\to V$ на эвклидовом или унитарном
пространстве $V$ называется \dfn{неотрицательно
  определенным}\index{оператор!неотрицательно определенный}, если
$B(a(v),v)\geq 0$ для любого $v\in V$. Оператор $a$ называется
\dfn{положительно
определенным}\index{оператор!положительно определенный}, если он
неотрицательно определен и из
$B(a(v),v)=0$ следует, что $v=0$.
\end{definition}

\begin{proposition}\label{prop:positive_definition}
Оператор $a\colon V\to V$ на эвклидовом или унитарном пространстве $V$
неотрицательно определен тогда и только тогда, когда в некотором
ортонормированном базисе матрица этого оператора диагональна, причем
на диагонали стоят неотрицательные вещественные числа.
Оператор $a$ положительно определен тогда и только тогда, когда в
некотором ортонормированном базисе матрица этого оператора
диагональна, причем на диагонали стоят положительные вещественные числа.
\end{proposition}
\begin{proof}
Если $a$ неотрицательно определен, то он (по определению)
самосопряжен, и по теоремам~\ref{thm:unitary_canonical_forms}
и~\ref{thm:euclidean_canonical_forms} существует ортонормированный
базис $\mc B = (e_1,\dots,e_n)$, в котором $a$ имеет
диагональную матрицу
$$
[a]_{\mc B} = \begin{pmatrix}\lambda_1 & 0 & \dots & 0 \\ 0 & \lambda_2 &
  \dots & 0\\ \vdots & \vdots & \ddots & \vdots \\ 0 & 0 & \dots &
  \lambda_n\end{pmatrix}.
$$
Предположим, что $\lambda_i<0$. Тогда $a(e_i) = \lambda_ie_i$ и
$B(a(e_i),e_i) = \lambda_i B(e_i,e_i) = \lambda_i < 0$, что
противоречит неотрицательной определенности $a$. Если же $a$
положительно определен, то и случай $\lambda_i=0$ невозможен: если
$\lambda_i=0$, то $B(a(e_i),e_i) = \lambda_i = 0$, в то время как
$e_i\neq 0$.

Обратно, пусть $a$ в некотором ортонормированном базисе $\mc
B=\{e_1,\dots,e_n\}$ имеет
диагональную матрицу с неотрицательными числами
$\lambda_1,\dots,\lambda_n$ на диагонали. По
теоремам~\ref{thm:unitary_canonical_forms}
и~\ref{thm:euclidean_canonical_forms} мы уже знаем, что $a$
самосопряжен. Разложим произвольный вектор $v$ по базису $\mc B$:
$v = \sum_i e_i c_i$.
Тогда $a(v) = \sum_i a(e_i) c_i = \sum_i e_i c_i\lambda_i$.
Поэтому
$$
B(a(v),v) = B(\sum_i e_i c_i\lambda_i,\sum_j e_j c_i)
= \sum_{i,j}\overline{c_i}\lambda_i c_j B(e_i,e_j)
= \sum_i\lambda_i \overline{c_i}c_i B(e_i,e_i)
= \sum_i\lambda_i |c_i|^2 \geq 0.
$$
Если же все $\lambda_i>0$ и оказалось, что $\sum_i\lambda_i
|c_i|^2=0$, то и $c_i=0$ для всех $i$, откуда $v=0$.
\end{proof}

\begin{remark}\label{rem:positive_invertible}
Таким образом, положительно определенный оператор всегда является
обратимым: его матрица в некотором базисе имеет
ненулевой определитель. Кроме того, если неотрицательно определенный
оператор обратим, то он положительно определен: у обратимой
диагональной матрицы не может встретиться $0$ на диагонали.
\end{remark}

\begin{theorem}[Извлечение квадратного корня в классе положительно
  определенных операторов]\label{thm:square_root_positive}
Пусть $a\colon V\to V$~--- положительно определенный
оператор на эвклидовом или унитарном пространстве $V$. Существует
единственный положительно определенный оператор
$b\colon V\to V$ такой, что $b^2 = a$.
\end{theorem}
\begin{proof}
По предложению~\ref{prop:positive_definition} найдется базис
$\mc B=(e_1,\dots,e_n)$, такой, что
$$
[a]_{\mc B} = \begin{pmatrix}\lambda_1 & 0 & \dots & 0 \\ 0 & \lambda_2 &
  \dots & 0\\ \vdots & \vdots & \ddots & \vdots \\ 0 & 0 & \dots &
  \lambda_n\end{pmatrix},
$$
причем $\lambda_i$~--- положительно вещественные числа. Рассмотрим
оператор $b$, матрица которого в базисе $\mc B$ равна
$$
[a]_{\mc B} = \begin{pmatrix}\sqrt{\lambda_1} & 0 & \dots & 0 \\ 0 & \sqrt{\lambda_2} &
  \dots & 0\\ \vdots & \vdots & \ddots & \vdots \\ 0 & 0 & \dots &
  \sqrt{\lambda_n}\end{pmatrix}.
$$
Заметим, что $\sqrt{\lambda_i}>0$ для всех $i$, поэтому (снова по
предложению~\ref{prop:positive_definition}) оператор $b$ положительно
определен. Кроме того, очевидно, что $b^2 = a$.

Нам осталось показать, что такой оператор $b$ единственный.
Пусть $\widetilde{b}$~--- другой оператор с теми же
свойствами: $\widetilde{b}$ положительно определен и $\widetilde{b}^2
= a$.
 Воспользуемся замечанием~\ref{rem:self_adjoint_geometry}
для оператора $\widetilde{b}$. А именно, пусть $\mu_1,\dots,\mu_n$~---
собственные числа оператора $\widetilde{b}$ с учетом кратности. Тогда
$\widetilde{b}$ приводится в некотором базисе к диагональному виду, и
на диагонали стоят положительные числа $\mu_1,\dots,\mu_n$. Но тогда $a =
\widetilde{b}^2$ в этом же базисе имеет диагональный вид, и на
диагонали стоят числа $\mu_1^2,\dots,\mu_n^2$. Значит, собственные
числа оператора $a$ (с учетом кратности) равны
$\mu_1^2,\dots,\mu_n^2$. С другой стороны, мы знаем, что они равны
$\lambda_1,\dots,\lambda_n$. Мы знаем, что $\mu_i>0$ для всех $i$,
поэтому набор $\mu_1,\dots,\mu_n$ совпадает (с точностью до
перестановки) с набором $\sqrt{\lambda_1},\dots,\sqrt{\lambda_n}$.

Мы получили, что наборы собственных чисел операторов $b$ и
$\widetilde{b}$ совпадают. Осталось показать, что собственные
подпространства для этих операторов, соответствующие одинаковым
собственным числам, совпадают, и воспользоваться соответствием из
замечания~\ref{rem:self_adjoint_geometry}.

Пусть теперь $V_i$~--- собственное подпространство для оператора $b$,
соответствующее собственному числу $\sqrt{\lambda_i}$. Оно натянуто на те
векторы базиса $\mc B$, которым соответствуют номера столбиков, в
которых в матрице $b$ стоят числа $\sqrt{\lambda_i}$. После возведения
в квадрат матрица остается диагональной, поэтому $V_i$ является
собственным подпространством оператора $a$, соответствующим
собственному числу $\lambda_i$. Но то же самое рассуждение применимо и
к оператору $\widetilde{b}$. Поэтому собственные подпространства для
операторов $b$ и $\widetilde{b}$, соответствующие $\sqrt{\lambda_i}$,
совпадают.
\end{proof}

Следующая теорема является прямым обобщением того факта, что
любое ненулевое комплексное число $z$ можно (единственным образом)
записать в
тригонометрической форме
(см. определение~\ref{dfn:trigonometric_form}):
$z = |z|\cdot (\cos(\ph)+i\sin(\ph))$.
Здесь
$|z|$~--- положительное вещественное число, а $(\cos(\ph) +
i\sin(\ph))$~--- комплексное число, которое по модулю равно
$1$. Полярное разложение обобщает эту теорему на многомерный случай:
слова <<ненулевое число>> нужно заменить на <<обратимый оператор>>,
слова <<положительное вещественное число>> на <<положительно
определенный оператор>>, а <<комплексное число, равное по модулю
$1$>>~--- на <<унитарный оператор>>. Обратите внимание, что матрица
$1\times 1$ задается ровно одним числом, поэтому при подстановке в
следующую теорему одномерного векторного пространства $V=\mb C$
действительно получается утверждение о тригонометрической форме
комплексного числа. Вещественный случай еще проще: если
$z\in\mb R\setminus\{0\}$, то $z = |z|\cdot(\pm 1)$; ортогональный
оператор на одномерном пространстве может быть равен лишь $1$ или
$-1$.

\begin{theorem}[Полярное разложение]\label{thm:polar_decomposition}
Пусть $a\colon V\to V$~--- обратимый оператор на эвклидовом или
унитарном пространстве. Тогда существуют операторы $p,u\colon V\to V$
такие, что $a = pu$, причем $p$~--- положительно определенный
оператор, а $u$~--- ортогональный или унитарный. Более того, такие
операторы единственны: если $a=p'u'$ для положительно определенного
$p$ и ортогонального/унитарного $u$, то $p=p'$ и $u=u'$.
\end{theorem}
\begin{proof}
Рассмотрим оператор $c = a\circ a^*$. Заметим, что $c$ самосопряжен:
действительно, $c^* = (a\circ a^*)^* = a^{**}\circ a^* = a\circ a^* =
c$.
Кроме того, $c$ неотрицательно определен:
$B(c(v),v) = B((a\circ a^*)(v),v) = B(a(a^*(v)),v) =
B(a^*(v),a^*(v))\geq 0$.
Наконец, поскольку $a$ обратим, то и $a^*$ обратим (их матрицы в
ортонормированном базисе транспонированны, поэтому из обратимости
одной следует обратимость другой), значит, и $c$ обратим; поэтому $c$
положительно определен (см. замечание~\ref{rem:positive_invertible}).
По теореме~\ref{thm:square_root_positive} из $c$ можно извлечь
квадратный корень: найдется положительно определенный оператор $p$
такой, что $p^2 = c = a\circ a^*$. В силу положительной определенности
оператор $p$ обратим.
Обозначим теперь $u = p^{-1}a$. Тогда, очевидно, $a = pu$, и осталось
проверить, что $u$~--- ортогональный/унитарный оператор.
Заметим сначала, что $pp^{-1} = \id$, поэтому
$(pp^{-1})^* = \id^* = \id$, откуда $(p^{-1})^* = p^{-1}$.
Поэтому $u\circ u^* = p^{-1}a(p^{-1}a)^* = p^{-1}aa^*(p^{-1})^* =
p^{-1}p^2 p^{-1} = \id$, что и требовалось.

Наконец, если $pu = a = p'u'$, то $(pu)^* = (p'u')^*$, откуда $u^* p =
(u')^*p'$. Из этого следует, что
$(pu)(u^*p) = (p'u')((u')^*p')$, откуда $p^2 = (p')^2$, и в силу
единственности извлечения квадратного корня
(теорема~\ref{thm:square_root_positive}), получаем, что
$p=p'$, и, стало быть, $u=u'$.
\end{proof}

\begin{remark}
Даже доказательство теоремы~\ref{thm:polar_decomposition}
 напоминает доказательство факта про
тригонометрическую форму записи комплексного числа: напомним, что
модуль комплексного числа $z$ определялся как $\sqrt{z\cdot\ol{z}}$
(см. определение~\ref{dfn:absolute_value_complex}); извлечение корня
возможно в силу неотрицательности $z\cdot\ol{z}$.
\end{remark}
